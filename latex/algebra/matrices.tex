\begin{indice}
\minitoc
\end{indice}

\newpage 

\section{Definiciones. Tipos de matrices}

\subsection{Definiciones}


\begin{definicion}
Una {\bf matriz de dimensión $m \times n$}  es un conjunto de números ordenados en m filas y n columnas de la forma:
 \[ A=\begin{pmatrix} 
 a_{11} & a_{12} & \cdots & a_{1n} \\
 a_{21} & a_{22} & \cdots & a_{2n} \\
  \vdots  & \vdots  & \cdots & \vdots \\
 a_{m1} & a_{m2} & \cdots & a_{mn} \\
\end{pmatrix}
\]
\end{definicion}

Cada elemento genérico de la matriz se designa por $a_{ij}$, donde $i$ es el número de fila que ocupa y $j$ es el número de fila.

\begin{definicion}[Igualdad]
Dos matrices de la misma dimensión $A_{mxn}=(a_{ij})$ y $B_{mxn}=(b_{ij})$ son iguales si $a_{ij}=b_{ij}\quad \forall i,j$
\end{definicion}

\begin{ejemplo}
Dada la matriz
$A= \begin{pmatrix}  -1 & 1  \\  2 & -3 \\  3 & -5 \\ \end{pmatrix}$ , escribir su dimensión y los elementos $a_{12}$, $a_{22}$ y $a_{31}$
\tcblower
La matriz tiene 3 filas y  2 columnas, luego su dimensión es $3x2$.

$a_{12} = 1$, $a_{22}= -3$ y $a_{31}=3$
\end{ejemplo}

\begin{ejemplo}
Hallar los valores de $x$ e $y$ para que las matrices $ \begin{pmatrix}  x & 1 \\  2 & y \\ \end{pmatrix}$, $ \begin{pmatrix}  3 & x-2 \\  y & x-1 \\ \end{pmatrix}$ sean iguales.
\tcblower
Para que sean iguales cada elemento de la primera matriz tiene que ser igual a su correspondiente de la segunda.

Por lo tanto, $\begin{cases} x= 3 \cr 1= x-2 \cr 2= y \cr y=x-1 \end{cases}$
 
Resolviendo tenemos que $x=3 $ e $ y=1$.
\end{ejemplo}

\subsection{Tipos de matrices}

\begin{enumerate}[label={\alph* )}]
	\item  {\bf Matriz fila}: es la matriz que solo tiene una fila $A_{1n}$.
 
  		$A= \begin{pmatrix}  -1 & 1  & 2  \\ \end{pmatrix}$
  
    \item  {\bf Matriz columna}: es la matriz que solo tiene una fila $A_{m1}$
   
  		$A= \begin{pmatrix}  -1 \\  1  \\  2  \\  -3   \\\end{pmatrix}$

  
   \item  {\bf Matriz nula}: es la matriz en la que todos sus elementos son nulos.
 
  $A= \begin{pmatrix}  0 & 0  \\  0  & 0 \\   0 & 0 \\\end{pmatrix}$

 	\item  {\bf Matriz rectangular}: es la matriz que tiene distinto número de filas que de columnas.
 
  		$A= \begin{pmatrix}  -1 & 1  & 2  \\  -3  & 3 & -5 \\\end{pmatrix}$

	\item {\bf Matriz cuadrada}: es la matriz que tiene igual número de filas que de columnas (m=n). En este caso podemos decir que la matriz es de orden n. 

  	$B= \begin{pmatrix}  -1 & 1  & 2  \\  -3  & 3 & -5 \\  2  & 1 & 6 \\\end{pmatrix}$


  	En una matriz cuadrada definimos:
  \begin{enumerate}[label={\roman* )}]
  \item {\bf Diagonal principal}: Son los elementos cuyo número de fila coincide con el de columna: $a_{ii}$
  \item {\bf Diagonal secundaria}: Son todos los elementos tales que $i+j=n+1$.
  \end{enumerate}
  
\end{enumerate}

\[ \begin{array}{cc}
\begin{tikzpicture}[baseline=(A.center)]
\tikzset{BarreStyle/.style =   {opacity=.4,line width=4 mm,line cap=round,color=red}}
  \matrix (A) [matrix of math nodes,ampersand replacement=\&,column sep=0 mm,left delimiter={(},right delimiter={)}] {a_{11} \& a_{12} \& \cdots \& a_{1n}  \\
  a_{21} \& a_{22}  \& \cdots \& a_{2n}  \\
  \cdots \& \cdots \& \vdots \& \vdots \\
  a_{n1} \& a_{n2}  \& \cdots \& a_{nn}  \\ };
\draw [BarreStyle] (A-1-1.north west)  to (A-4-4.south east) ;
\end{tikzpicture} &  \begin{tikzpicture}[baseline=(A.center)]
\tikzset{BarreStyle/.style =   {opacity=.4,line width=4 mm,line cap=round,color=blue}}
  \matrix (A) [matrix of math nodes,ampersand replacement=\&,column sep=0 mm,left delimiter={(},right delimiter={)}] {a_{11} \& a_{12} \& \cdots \& a_{1n}  \\
  a_{21} \& a_{22}  \& \cdots \& a_{2n}  \\
  \cdots \& \cdots \& \vdots \& \vdots \\
  a_{n1} \& a_{n2}  \& \cdots \& a_{nn}  \\ };
\draw [BarreStyle] (A-1-4.north east)  to (A-4-1.south west) ;
\end{tikzpicture} \\
 \text{Diagonal principal}& \text{Diagonal secundaria}
\end{array} \]

\vspace{1cm}
Dentro de las matrices cuadradas podemos distinguir las siguientes:

 \textbf{Matrices triangular superior:} Son las que todos sus elementos por debajo de la diagonal principal son cero.

\[ \begin{pmatrix} 
1 & -1 & 3 & 0  \\
  \color{red}0 & 3  & 1 & -1  \\
 \color{red}0 & \color{red}0 & 2 & 1 \\
  \color{red}0 & \color{red}0  & \color{red}0 & -4  \\ 
\end{pmatrix} \]

\textbf{Matrices triangular inferior:} Son las que todos sus elementos por encima de la diagonal principal son cero.

\[
  \begin{pmatrix}1 & \color{red}0 & \color{red}0 & \color{red}0  \\
  -1 & 3  & \color{red}0 & \color{red}0  \\
  1 & 2 & -2 & \color{red}0 \\
 4 & 0  & 6 & -2  \\ 
\end{pmatrix} \]

\textbf{Matrices diagonal:} Son las que todos sus elementos fuera de la diagonal principal son cero.

\[
  \begin{pmatrix} 1 & \color{red}0 & \color{red}0 & \color{red}0  \\
  \color{red}0 & 3  & \color{red}0 & \color{red}0  \\
  \color{red}0 & \color{red}0 & -2 & \color{red}0 \\
 \color{red}0 & \color{red}0  & \color{red}0 & -2  \\ 
\end{pmatrix}
\] 

 \textbf{Matrices escalar:} Son las que todos sus elementos  de la diagonal principal son iguales y el resto cero.

\[
  \begin{pmatrix} -2 & \color{red}0 & \color{red}0 & \color{red}0  \\
  \color{red}0 & -2  & \color{red}0 & \color{red}0  \\
  \color{red}0 & \color{red}0 & -2 & \color{red}0 \\
 \color{red}0 & \color{red}0  & \color{red}0 & -2  \\ 
\end{pmatrix}
\] 

\textbf{Matrices identidad:} Son las que todos sus elementos  de la diagonal principal son 1 y el resto cero.

\[
  \begin{pmatrix} 1 & \color{red}0 & \color{red}0 & \color{red}0  \\
  \color{red}0 & 1  & \color{red}0 & \color{red}0  \\
  \color{red}0 & \color{red}0 & 1 & \color{red}0 \\
 \color{red}0 & \color{red}0  & \color{red}0 & 1  \\ 
\end{pmatrix}
\] 

La matriz identidad se llama $I_n$, siendo $n$ el orden de la matriz.

\section{Operaciones con matrices}


\begin{definicion}[Suma]
Dadas dos matrices de la misma dimensión $A=(a_{ij})$ y $B=(b_{ij})$ llamamos suma a otra matriz $C=(c_{ij})$ definida por 
$$(c_{ij})=(a_{ij})+(b_{ij})=(a_{ij}+b_{ij})$$

Y cumple las siguientes propiedades
\begin{multicols}{2}
	\begin{enumerate}
		\item $A+B=B+A$
		\item $A+(B+C)=(A+B)+C$
		\item$A+0=A$
		\item $A-A=0$
	\end{enumerate}
	\end{multicols}
\end{definicion}

\begin{ejemplo}
Sumar $$   \begin{pmatrix}
	2 & 1 \\
	1 & 1 \\
	3 & 2
\end{pmatrix} 
     +
     \begin{pmatrix}
	1 & 0 \\
	-1 & -2 \\
	2 & 3
\end{pmatrix}    $$
\tcblower
$$   \begin{pmatrix}
	2 & 1 \\
	1 & 1 \\
	3 & 2
\end{pmatrix} 
     +
     \begin{pmatrix}
	1 & 0 \\
	-1 & -2 \\
	2 & 3
\end{pmatrix}    = \begin{pmatrix}
	3 & 1 \\
	0 &-1 \\
	5 & 5
\end{pmatrix}    $$
\end{ejemplo}


\begin{definicion}[Producto por un número real]
Dados un número real $k\in \R$ y una matriz $A=(a_{ij})$ se define el producto de $k\cdot A$ de la siguiente manera:\\
$$k\cdot A=k\cdot (a_{ij})= (k\cdot a_{ij})$$
 $$5\cdot\begin{pmatrix}
	2 & 1 \\
	1 & 1 \\
	3 & 2
\end{pmatrix}= \begin{pmatrix}
	10 & 5 \\
	5 & 5 \\
	15 & 10
\end{pmatrix}  $$
\end{definicion}

\begin{definicion}[Producto de matrices]
Dos matrices $A$ y $B$ se pueden multiplicar si el número de columnas de $A$ es igual al número de filas de $B$, en tal caso, se define la multiplicación de la siguiente manera
$${A \atop m\ast n}\cdot {B \atop n\ast p}$$
$$\underbrace{A}_{m \ast n} \cdot \underbrace{B}_{n\ast p}=\underbrace{C}_{m\ast p}$$
Para calcular el elemento $c_{ij}$ hay que utilizar la fila $i$ de $A$ y la columna $j$ de B.
$$\begin{pmatrix}
	a_{i1} & a_{i2}&a_{i3}& \cdots &a_{in} \\
\end{pmatrix}\cdot \begin{pmatrix}
	a_{j1} \\ a_{j2}\\a_{j3}\\ \cdots \\a_{jn} \\
\end{pmatrix}=a_{i1}\cdot a_{j1}+ a_{i2} \cdot a_{2j} + \cdots +a_{in}\cdot a_{nj}$$
$$\begin{pmatrix}
	2 & 1 \\
	1 & 1 \\
	3 & 2
\end{pmatrix}\cdot \begin{pmatrix}
	10 & 5 &5\\
	5 & 5 &-2
\end{pmatrix}=\begin{pmatrix}
	2\cdot 10 + 1 \cdot 5 & 2\cdot 5 + 1 \cdot 5 & 2\cdot 5 + 1 \cdot (-2)\\
	1\cdot 10 + 1 \cdot 5 & 1\cdot 5 + 1 \cdot 5 & 1\cdot 5 + 1 \cdot (-2)\\
	3\cdot 10 + 2 \cdot 5 & 3\cdot 5 + 2 \cdot 5 & 3\cdot 5 + 2 \cdot (-2)
\end{pmatrix} = \begin{pmatrix}
	25 & 15& 8)\\
	15 & 10 & 3\\
	40 & 25 & 11
\end{pmatrix}  $$\\
\textbf{{\Large El producto de matrices no es conmutativo $A \cdot B \neq B\cdot A$}}

\end{definicion}

\begin{ejemplo}
Dadas las matrices 
\[A=\begin{pmatrix}
	2&2&-1 \\
	-1&-1&1 \\
	-1&-2&2
\end{pmatrix}
\qquad
I=\begin{pmatrix}
	1&0&0 \\
	0&1&0 \\
	0&0&1
\end{pmatrix} \]

se pide:

 Calcular la matriz $(A-I)^2$

\tcblower

 Primero calculamos $A-I=\begin{pmatrix}
	2&2&-1 \\
	-1&-1&1 \\
	-1&-2&2
\end{pmatrix}
-\begin{pmatrix}
	1&0&0 \\
	0&1&0 \\
	0&0&1
\end{pmatrix}=
\begin{pmatrix}
	1 & 2 & -1 \\
	-1&-2&1 		\\
	-1&-2 & 1
\end{pmatrix} $ 

Ahora calculamos $(A-I)^2=\begin{pmatrix}
	1 & 2 & -1 \\
	-1&-2&1 		\\
	-1&-2 & 1
\end{pmatrix} \cdot\begin{pmatrix}
	1 & 2 & -1 \\
	-1&-2&1 		\\
	-1&-2 & 1
\end{pmatrix} =
\begin{pmatrix}
	0 & 0 & 0 \\
	0&0&0 		\\
	0&0 & 0
\end{pmatrix} $

\end{ejemplo}

\begin{ejemplo}
Sea $A=\begin{pmatrix}
1 & 0 \\
-1 & 1
\end{pmatrix}$. Se pide:
\begin{enumerate}[label={\alph* ) }]
\item Demostrar que $A^2=2A-I$, donde $I$ es la matriz identidad de orden 2.
\item Expresar $A^3$ y $A^4$ en función de $A$.
\item Calcular $A^{100}$
\end{enumerate}


\tcblower

\begin{enumerate}[label={\alph* ) }]
\item Calculamos $A^2$

$A^2=\begin{pmatrix}
1 & 0 \\
-1 & 1
\end{pmatrix} \cdot\begin{pmatrix}
1 & 0 \\
-1 & 1
\end{pmatrix}= \begin{pmatrix}
1 & 0 \\
-2 & 1
\end{pmatrix}$

Ahora calculamos $2A-I=2\begin{pmatrix}
1 & 0 \\
-1 & 1
\end{pmatrix}-
\begin{pmatrix}
1 & 0 \\
0 & 1
\end{pmatrix}=\begin{pmatrix}
1 & 0 \\
-2 & 1
\end{pmatrix}$

Como vemos son iguales.

\item

$A^3=A^2\cdot A= (2A-I)\cdot A= 2A^2-I\cdot A=2A^2-A=2(2A-I)-A=3A-2I$

$A^4=A^3\cdot A=(3A-2I)\cdot A=3A^2-2I\cdot A= 3(2A-I)-2A=4A-3I$ 
\item 

$A^{100}=100A-99I=100\begin{pmatrix}
1 & 0 \\
-1 & 1
\end{pmatrix}-
99\begin{pmatrix}
1 & 0 \\
0 & 1
\end{pmatrix}=
\begin{pmatrix}
1 & 0 \\
-100 & 1
\end{pmatrix}$
\end{enumerate}

\end{ejemplo}

\begin{ejemplo}
Dada la matriz $A=\begin{pmatrix}
 0 & 0 & 1 \\
 0 & a & 0 \\
 -1 & 0 & -2 
\end{pmatrix}
 $, se pide:

Determínese el valor o los valores del parámetro $a$ para que se verifique que $A^2+2A+I=O$, siendo $I$ la matriz unidad y $O$ la matriz nula, ambas de orden 3.

\tcblower

Calculamos $A^2$
\[ A^2=\begin{pmatrix}
 0 & 0 & 1 \\
 0 & a & 0 \\
 -1 & 0 & -2 
\end{pmatrix}\cdot
\begin{pmatrix}
 0 & 0 & 1 \\
 0 & a & 0 \\
 -1 & 0 & -2 
\end{pmatrix}= 
\begin{pmatrix}
 -1 & 0 & -2 \\
 0 & a^2 & 0 \\
 2 & 0 & 3 
\end{pmatrix}\]

Ahora calculamos $A^2+2A+I$
\[ \begin{pmatrix}
 -1 & 0 & -2 \\
 0 & a^2 & 0 \\
 2 & 0 & 3 
\end{pmatrix} +
2\begin{pmatrix}
 0 & 0 & 1 \\
 0 & a & 0 \\
 -1 & 0 & -2 
\end{pmatrix}+
\begin{pmatrix}
1 & 0 & 0 \\
0 & 1 & 0 \\
0 & 0 & 1
\end{pmatrix}=
\begin{pmatrix}
 0 & 0 & 0 \\
 0 & a^2+2a+1 & 0 \\
 0 & 0 & 0 
\end{pmatrix} \]

Igualando a la matriz nula:

\[
\begin{pmatrix}
0 & 0 & 0 \\
 0 & a^2+2a+1 & 0 \\
 0 & 0 & 0 
\end{pmatrix}=
\begin{pmatrix}
0 & 0 & 0 \\
0 & 0 & 0 \\
0 & 0 & 0
\end{pmatrix}
\]

Luego: $a^2+2a+1=0 \Rightarrow a=-1$

\end{ejemplo}

\begin{ejemplo}
Sean las matrices 
$A=\begin{pmatrix}
	2 & 1 \\
	1 & 1
\end{pmatrix}$, 
$B=\begin{pmatrix}
 	1 & x \\ 
 	x & 0
\end{pmatrix}$ y  
$C=\begin{pmatrix}
	0 & -1 \\
	-1 & 2
\end{pmatrix}
$

\begin{enumerate}[label=\alph*)]
\item Encuentre el valor o los valores de $x$ de forma que $B^2=A$
\item Determine $x$ para que $A+B+C=3I_2$
\end{enumerate}

\tcblower

\begin{enumerate}[label=\alph*) ]
\item Hallamos $B^2$. 
\[ \begin{pmatrix}
	1 & x \\
	x & 0 
\end{pmatrix} \cdot 
\begin{pmatrix}
	1 & x \\
	x & 0 
\end{pmatrix}=
\begin{pmatrix}
	x^2+1 & x \\
	x & x^2
\end{pmatrix}
\]

Igualamos las dos matrices : 
\[ \begin{pmatrix}
	x^2+1 & x \\
	x & x^2
\end{pmatrix} =
\begin{pmatrix}
	2 & 1 \\
	1 & 1
\end{pmatrix} \]
De aquí obtenemos el siguiente sistema de ecuaciones:
\[
\left\lbrace
\begin{array}{l}
x^2+1=2 \\
x=1 \\
x=1 \\
x^2=1
\end{array}
\right. \Rightarrow x=1 \]

\item Hallamos $A+B+C$

\[ 
\begin{pmatrix}
	2 & 1 \\
	1 & 1
\end{pmatrix} +  
\begin{pmatrix}
 	1 & x \\ 
 	x & 0
\end{pmatrix}
+\begin{pmatrix}
	0 & -1 \\
	-1 & 2
\end{pmatrix}=
\begin{pmatrix}
	3 & x \\
	x & 3
\end{pmatrix}
\]

Igualando esta matriz a $3I_2$ obtenemos:
\[
\begin{pmatrix}
	3 & x \\
	x & 3
\end{pmatrix}=
\begin{pmatrix}
	3 & 0 \\
	0 & 3
\end{pmatrix}
 \Rightarrow \left\lbrace
\begin{array}{l}
3=3 \\
x=0 \\
x=0 \\
3=3
\end{array}
\right. \Rightarrow x=0 \]

\end{enumerate}

\end{ejemplo}

\begin{ejer}
 Dadas  las matrices

$A=\begin{pmatrix}
 2 & 1 \\
 1 &  2 \\
 0 & 1
\end{pmatrix} $
, 
$B=\begin{pmatrix}
 2 & 0 & 1 \\
 3 & 2 & -1 \\
 1 & 1 & 1
\end{pmatrix} $,
 $C=\begin{pmatrix}
 3 & 2 & 1 \\
 -2 & -1 & 0
\end{pmatrix} $, 
$D=\begin{pmatrix}
 5 & 1 \\
 1 & 3
\end{pmatrix} $.

Indicar todos los posibles productos entre ellas y calcular el elemento (2,1) de cada producto.

\begin{solu}

Productos posibles:
$\begin{cases} 
A \cdot C & (A \cdot C)_{21}=-1 \\ 
A \cdot D & (A \cdot D)_{21}=7 \\ 
B \cdot A & (B \cdot A)_{21}=8 \\
C \cdot A & (C \cdot A)_{21}=-5 \\
D \cdot C & (D \cdot C)_{21}=-3 
\end{cases} $

\end{solu}
\end{ejer}

\begin{ejer}
Dadas las matrices:
$A=\begin{pmatrix}
 a & 2 \\
 1 & b
\end{pmatrix} $, 
$B=\begin{pmatrix}
 1 & 1 \\
 1 & 2
\end{pmatrix} $, 
$C=\begin{pmatrix}
 -1 \\
 1 
\end{pmatrix} $, hallar:
\begin{enumerate}[label=\alph*)]
\item Las matrices $BAC$ y $A^tC$ (donde $A^t$ es la traspuesta de $A$).
\item Los valores que deben tener $a$ y $b$ para que se cumpla $BAC=A^tC$.
\end{enumerate}
\begin{solu}
\begin{enumerate}[label=\alph*)]
\item $B\cdot A \cdot C= \begin{pmatrix} -a+b+1 \\ 2b-a \end{pmatrix} \qquad A^t \cdot C= \begin{pmatrix} -a+1 \\ b-2 \end{pmatrix}$
\item $a=2 \qquad b=0$
\end{enumerate}
\end{solu}
\end{ejer}

\begin{ejer}
Sea $A$ la matriz $A=\begin{pmatrix}
1 & a \\
0 & 1
\end{pmatrix}$

Para cada número natural $a$, hallar $A^n$. Calcular $A^{22}-2A^2+2A$
\begin{solu}

$A^n= \begin{pmatrix}
1 & na \\
0 & 1
\end{pmatrix} \qquad A^{22}-2A^2+2A=\begin{pmatrix}
1 & 20a \\
0 & 1
\end{pmatrix}$
\end{solu}
\end{ejer}

\begin{ejer}
Sean A y B la matrices dadas por:
\[A=\begin{pmatrix}
0 & 1 & 1 \\
1 & 1 & 0 \\
1 & 0 & 0
\end{pmatrix}
\qquad
B=\begin{pmatrix}
6 & -3 & -4 \\
-3 & 2 & 1 \\
-4 & 1 & 5
\end{pmatrix} \]
Contestar razonadamente a la siguiente pregunta. ¿Existe algún valor de $\lambda \in \R$ tal que la igualdad $(A-\lambda I)^2=B$ sea cierta?. En caso afirmativo hallar dicho valor de $\lambda $. $I$ es la matriz identidad de orden 3. 
\begin{solu}
$a=2$
\end{solu}
\end{ejer}

\begin{ejer}
 Sean $A$ y $M$ las matrices:
\[ A=\begin{pmatrix}
3 & 2 \\
1 & 3
\end{pmatrix}
\qquad
M=\begin{pmatrix}
m & n \\
p & q
\end{pmatrix} \]
Encontrar las condiciones que deben cumplir $m$, $n$, $p$ y $q$ para que se verifique que el producto de ambas matrices efectuado en las dos formas posibles sea el mismo.
\begin{solu}
$ m= q \qquad n=2p$
\end{solu}
\end{ejer}

\begin{ejer}
 Dadas las matrices 
\[ A=\begin{pmatrix}
5 & 2 & 0 \\
2 & 5 & 0 \\
0 & 0 & 1
\end{pmatrix}
\qquad
B=\begin{pmatrix}
a & b & 0 \\
c & c & 0 \\
0 & 0 & 1
\end{pmatrix}\]
se pide:

\begin{enumerate}[label=\alph*)]
\item Encontrar las condiciones que deben cumplir $a$, $b$ y $c$ para que se verifique $A\cdot B = B \cdot A$
\item Para $a=b=c=1$, calcular $B^{10}$
\end{enumerate}
\begin{solu}
\begin{enumerate}[label=\alph*)]
\item $a=b=c$
\item $\begin{pmatrix} 512 & 512 & 0 \\ 512 & 512 & 0 \\ 0 & 0  1 \end{pmatrix}$
\end{enumerate}

\end{solu}
\end{ejer}

\begin{ejer} Sea la matriz 
$A=\begin{pmatrix}
5 & -4 & 2 \\ 
 2 & -1 & 1 \\ 
-4 & 4 & -1
\end{pmatrix}
$

\begin{enumerate}[label=\alph*)]
\item Prueba que $A^2-2A+I=0$ donde $I$ es la matriz identidad y $0$ es una matriz con todos sus elementos iguales a 0.
\item Calcula $A^3$
\end{enumerate}
\begin{solu}
\begin{enumerate}[label=\alph*)]
\item Operar para demostrar.
\item $A^3=\begin{pmatrix} 13&-12&6 \\ 6&-5&3 \\ -12&12&-5 \end{pmatrix} $
\end{enumerate}

\end{solu}
\end{ejer}

\begin{ejer} Sea la matriz 
$B=\begin{pmatrix}
1 & 0 \\
1 & b
\end{pmatrix}$. 
Calcule el valor de $b$ para que $B^2=I_2$
\begin{solu}
$b=-1$
\end{solu}
\end{ejer}

\newpage

\section{Determinantes}

\begin{definicion}{Determinante de orden 2}
Dada una matriz de orden 2,$A=\begin{pmatrix}
a_{11}& a_{12}  \\
a_{21} & a_{22} 
\end{pmatrix}$, se llama determinante de $A$ al número real : 

\[ det(A)= |A|= \begin{vmatrix}
a_{11}& a_{12}  \\
a_{21} & a_{22} 
\end{vmatrix}=a_{11} \cdot a_{22}-a_{12}\cdot a_{21} \]
\end{definicion}

\begin{ejemplo}
   $\begin{vmatrix}
1& 3  \\
2 &5 
\end{vmatrix}=1\cdot 5-3\cdot 2=-1 $
\end{ejemplo}

\begin{definicion}{Determinante de orden 3}
Dada una matriz de orden 3,$A=\begin{pmatrix}
a_{11}& a_{12} & a_{13} \\
a_{21} & a_{22} & a_{23} \\
a_{31}& a_{32} & a_{33}
\end{pmatrix}$, se llama determinante de $A$ al número real: 

\[ det(A)= |A|= \begin{vmatrix}
a_{11}& a_{12}  &a_{13} \\
a_{21} & a_{22}&a_{23}\\
a_{31} & a_{32}&a_{33}
\end{vmatrix} = \]
\[ =a_{11} \cdot a_{12} \cdot a_{13}+a_{12} \cdot a_{23} \cdot a_{31}+a_{13} \cdot a_{32} \cdot a_{21}-a_{13} \cdot a_{22} \cdot a_{31}-a_{23} \cdot a_{32} \cdot a_{21}-a_{33} \cdot a_{21} \cdot a_{12} \]
\end{definicion}

Para recordar estos productos podemos utilizar la regla de Sarrus:

\begin{figure}[h]
\centering
\includegraphics[scale=0.7]{images/sarrus.jpg}
\end{figure}


\begin{ejemplo}
Calcular el siguiente determinante:
$\begin{vmatrix}1&2&3\\
0&-1&2\\
2&1&3
\end{vmatrix}$

\tcblower
Aplicamos la regla de Sarrus:

$\begin{vmatrix}1&2&3\\
0&-1&2\\
2&1&3
\end{vmatrix}$ $=1\cdot (-1) \cdot 3$ $ +2\cdot 2 \cdot 2$ $+ 3\cdot 1 \cdot 0 $  $-3\cdot (-1)\cdot 2 $ $- 2\cdot 0\cdot 3 $ $- 1\cdot 1 \cdot 2$ $=9$

\end{ejemplo}

\section{Propiedades de los determinantes}

\begin{enumerate}
\item Estas propiedades son válidas para determinantes de cualquier orden pero en los ejemplos usaremos determinantes de orden$3$

\item El determinante de una matriz y el de su traspuesta son iguales. Las propiedades valen indistintamente para filas o columnas.
\[ \vert A \vert =\vert A^t \vert \]

\item Si todos los elementos de una linea de un determinante son ceros, el determinante también es cero.
\[ \begin{vmatrix}
1&0&2\\
2&0&4\\
-1&0&3
\end{vmatrix}=0 \]

\item Si en un determinante se intercambian dos lineas, el determinante cambia de signo.
\[ \begin{vmatrix}
3&-2&5\\
1&7&3\\
4&1&0
\end{vmatrix}=-168\longrightarrow \begin{vmatrix}
1&7&3\\
3&-2&5\\
4&1&0
\end{vmatrix}=-(-168)=168 \]

\item Si en un determinante dos lineas son iguales, el determinante es cero.
\[ \begin{vmatrix}
1&2&1\\
3&-5&3\\
-2&6&-2
\end{vmatrix}=0 \].

\item Si los elementos de una linea se multiplican por un número, el determinante también se multiplica por ese número.
\[ \begin{vmatrix}
1&5\cdot 7&0\\
-2&3\cdot 7 &4\\
3&7\cdot 7& -1
\end{vmatrix}=7\begin{vmatrix}
1&5&0\\
-2&3 &4\\
3&7& -1
\end{vmatrix} \]

\item Si en un determinante dos lineas son proporcionales, el determinante es cero.
\[ \begin{vmatrix}
1&5&0\\
-2&3 &4\\
2&10& 0
\end{vmatrix} \stackrel{F_{3}=2 \cdot F_{1}}{=}0 \]

\item Si los elementos de una línea se pueden descomponer en dos sumandos,  su determinante es igual a la suma de dos determinantes que tienen iguales todas las líneas excepto la línea cuyos sumandos pasan, respectivamente, a cada uno de los determinantes. 
\[ \begin{vmatrix}
a_{11}& a_{12}+ a'_{12} &a_{13} \\
a_{21} & a_{22}+a'_{22}&a_{23}\\
a_{31} & a_{32}+a'_{32}&a_{33}
\end{vmatrix}=\begin{vmatrix}
a_{11}& a_{12} &a_{13} \\
a_{21} & a_{22}&a_{23}\\
a_{31} & a_{32}&a_{33}
\end{vmatrix}+\begin{vmatrix}
a_{11}&  a'_{12} &a_{13} \\
a_{21} & a'_{22}&a_{23}\\
a_{31} & a'_{32}&a_{33}
\end{vmatrix} \]

\item Si a una linea de un determinante se le suma una combinación lineal de las demás lineas, el valor del determinante no varía.
\[ \begin{vmatrix}
1&5&1\\
-2&3 &4\\
2&10& 3
\end{vmatrix} \stackrel{C_{2}+C_{1}-C_{3}}{=}\begin{vmatrix}
1&5&1\\
-2&-3 &4\\
2&9& 3
\end{vmatrix} \]

\item Si en una determinante una linea es combinación lineal de las demás el valor del determinante es cero.
\[ \begin{vmatrix}
1&3&4\\
2&3 &5\\
2&-3&-1
\end{vmatrix} \stackrel{C_{3}=C_{1}+C_{2}}{=}0 \]

\item El determinante del producto de dos matrices cuadradas es igual al producto de los determinantes de ambas matrices 
\[ \vert A \cdot B\vert=\vert A \vert\cdot \vert B\vert  \]

\end{enumerate}

\begin{ejemplo}
 Sabiendo que 
 $ \begin{vmatrix}
 a & b & c \\
 1 & 1 & 1 \\
 3 & 0 & 1
 \end{vmatrix}= 2 $
 calcula, usando las propiedades de los determinantes,
$ \begin{vmatrix}
 3-a & -b & 1-c \\
 1+a & 1+b & 1+c \\
 3a & 3b & 3c 
 \end{vmatrix}
 \qquad \text{y} \qquad 
 \begin{vmatrix}
  2a & 2b & 2c \\
 30 & 0 & 10 \\
  4 & 4 & 4
 \end{vmatrix} $

\tcblower
$
\begin{vmatrix}
 3-a & -b & 1-c \\
 1+a & 1+b & 1+c \\
 3a & 3b & 3c 
 \end{vmatrix}
 = 
 \begin{vmatrix}
 3 & 0 & 1 \\
 1+a & 1+b & 1+c \\
 3a & 3b & 3c 
 \end{vmatrix}+\begin{vmatrix}
 -a & -b & -c \\
 1+a & 1+b & 1+c \\
 3a & 3b & 3c 
 \end{vmatrix} =  
 \begin{vmatrix}
 3 & 0 & 1 \\
 1+a & 1+b & 1+c \\
 3a & 3b & 3c 
 \end{vmatrix} +  0 (F_3=-3F_1)
 = 
 \begin{vmatrix}
 3 & 0 & 1 \\
 1 & 1 & 1 \\
 3a & 3b & 3c 
 \end{vmatrix}+ \begin{vmatrix}
 3 & 0 & 1 \\
 a & b & c \\
 3a & 3b & 3c 
 \end{vmatrix} 
\begin{vmatrix}
 3 & 0 & 1 \\
 1 & 1 & 1 \\
 3a & 3b & 3c 
 \end{vmatrix}+  0 (F_3=3F_2)
  = 3\begin{vmatrix}
 3 & 0 & 1 \\
 1 & 1 & 1 \\
 a & b & c 
 \end{vmatrix}
= -3 \begin{vmatrix}
 a & b & c\\
 1 & 1 & 1 \\
   3 & 0 & 1
 \end{vmatrix} = -6 $

$
\begin{vmatrix}
  2a & 2b & 2c \\
  30 & 0 & 10 \\
 4 & 4 & 4
 \end{vmatrix} =  2\cdot 10 \cdot 4 \begin{vmatrix}
 a & b & c \\
 3 & 0 & 1 \\
 1 & 1 & 1
 \end{vmatrix} = 800  
$

\end{ejemplo}

\section{Matriz adjunta}

\subsection{Menor complementario y adjunto}


\begin{definicion}{Menor complementario}
Dada una matriz cuadrada de orden $n$, $A=(a_{ij})$, se llama menor complementario del elemento $a_{ij}$ al determinante de orden $n-1$ que se obtiene al eliminar la fila $i$ y la columna $j$ de la matriz $A$. Se suele representar por $M_{ij}$.
\end{definicion}

\begin{ejemplo}
Dada la matriz $A=\begin{pmatrix} 
1 & -2 & 3 \\
-1 & 0 & 2 \\
-2 & 3 & -1 \\
\end{pmatrix}$, hallar los menores complementarios $M_{12}$, $M_{22}$ y $M_{31}$
\tcblower

Para hallar los menores complementarios formamos los determinantes de orden $2$, ya que la matriz es de orden $3$, que resultan de eliminar la fila y la columna correspondiente.

\[ 
M_{12}=\begin{vmatrix}  -1 & 2 \\ -2 & -1 \\ \end{vmatrix}=5 \]
\[ M_{22}=\begin{vmatrix}  1 & 3 \\ -2 & -1 \\ \end{vmatrix}=4 \]
\[M_{31}=\begin{vmatrix}  -2 & 3 \\ 0 & 2 \\ \end{vmatrix}=-4
\]
\end{ejemplo}

\begin{definicion}{Adjunto}
Dada una matriz cuadrada de orden $n$, $A=(a_{ij})$, se llama adjunto del elemento $a_{ij}$ al menor complementario de dicho elemento precedido del signo $+$ o $-$, según si la suma de $i+j$ es par o impar. Se suele representar por $A_{ij}$.
\[
A_{ij}=(-1)^{i+j} \cdot M_{ij}
\]
\end{definicion}

\begin{ejemplo}
Dada la matriz $A=\begin{pmatrix} 
 1& -2& 3 \\
 -1& 0& 2 \\
 -2& 3& -1 \\
\end{pmatrix}$, hallar los adjuntos $A_{12}$, $A_{22}$ y $A_{31}$

\tcblower
Para hallar los adjuntos anteponemos el signo $+$ o $-$ a los menores correspondientes (calculados en el ejemplo anterior).

\[
A_{12}=(-1)^{1+2} \cdot M_{12}=-5 \qquad A_{22}=(-1)^{2+2} \cdot M_{22}=4 \qquad A_{13}=(-1)^{1+3} \cdot M_{31}=-4
\]
\end{ejemplo}

\subsection{Matriz adjunta}


\begin{definicion}{Matriz adjunta}
Se llama matriz adjunta de una matriz cuadrada $A(a_{ij})$ a la matriz que resulta de cambiar cada elemento $a_{ij}$ de la matriz por el adjunto de ese elemento $A_{ij}$. Se representa por $Adj(A)$.
\end{definicion}

\begin{ejemplo}
Dada la matriz $A=\begin{pmatrix} 
 1& -2& 3 \\
 -1& 0& 2 \\
 -2& 3& -1 \\
\end{pmatrix}$, hallar su matriz adjunta
\tcblower
Calculamos el adjunto de cada elemento.

\[
\begin{aligned}
A_{11}= + \begin{vmatrix} 0& 2 \\ 3& -1 \\ \end{vmatrix} =   -6 \quad & 
A_{12}= - \begin{vmatrix} -1& 2 \\ -2& -1 \\ \end{vmatrix} =   -5 & 
A_{13}=  + \begin{vmatrix} -1& 0 \\ -2& 3 \\ \end{vmatrix} =   -3 \\
A_{21}= - \begin{vmatrix} -2& 3 \\ 3& -1 \\ \end{vmatrix} =   7 \quad &  
A_{22}= + \begin{vmatrix} 1& 3 \\ -2& -1 \\ \end{vmatrix} =   5 &  
A_{23}= - \begin{vmatrix} 1& -2 \\ -2& 3 \\ \end{vmatrix} =   1 \\
A_{31}= + \begin{vmatrix} -2& 3 \\ 0& 2 \\ \end{vmatrix} =   -4 \quad &
A_{32}=- \begin{vmatrix} 1& 3 \\ -1& 2 \\ \end{vmatrix} =   -5 &
A_{33} + \begin{vmatrix} 1& -2 \\ -1& 0 \\ \end{vmatrix} =   -2 \\
\end{aligned}
\]

Para obtener la matriz adjunta cambiamos cada elemento por su adjunto.

\[
Adj(A)=
\begin{pmatrix}
  -6 &  -5  &  -3 \\
  7 &  5&  1 \\
  -4&  -5&  -2 \\
\end{pmatrix}
\]
\end{ejemplo}


\section{Desarrollo de un determinante por adjuntos}

\begin{definicion}{Desarrollo determinantes por adjuntos}
El determinante de una matriz cuadrada de orden $n$ es igual a la suma de los productos de cada elemento de esa fila por sus adjuntos respectivos.

$det(A)= a_{i1}\cdot A_{i1}+a_{i2}\cdot A_{i2}+\cdots +a_{in}\cdot A_{in}$

$det(A)= a_{1j}\cdot A_{1j}+a_{2j}\cdot A_{2j}+\cdots +a_{nj}\cdot A_{nj}$
\end{definicion}



\begin{ejemplo}
Hallar el determinante de la matriz $A=\begin{pmatrix} 
 1& -2& 3 \\
 5& 0& 6 \\
 -1& 2& -4 \\
\end{pmatrix} $
\begin{enumerate}[label=\alph* )]
\item Por la regla de Sarrus.
\item Desarrollando por la primera fila.
\item Desarrollando por la segunda columna.
\end{enumerate}

\tcblower
\begin{enumerate}[label=\alph* )]
\item $det(A)= 0+12+30-0-12-40=-10$
\item $det(A)=1\cdot \begin{vmatrix}& 0& 6 \\& 2& -4 \\ \end{vmatrix} -(-2)\cdot \begin{vmatrix}& 5& 6 \\& -1& -4 \\ \end{vmatrix} +3\cdot \begin{vmatrix}& 5& 0 \\& -1& 2 \\ \end{vmatrix}= -12-28+30=-10$
\item $det(A)=-(-2)\cdot \begin{vmatrix}& 5& 6 \\& -1& -4 \\ \end{vmatrix} +0\cdot \begin{vmatrix}& 1& 3 \\& -1& -4 \\ \end{vmatrix} -2\cdot \begin{vmatrix}& 1& 3 \\& 5& 6 \\ \end{vmatrix}= -28+0+18=-10$

En la practica, antes de calcular el determinante, hacemos cero el mayor número posible de elementos de una fila,utilizando las propiedades de los determinantes, y desarrollamos por los elementos de esa fila.

\end{enumerate}

\end{ejemplo}

\begin{ejemplo}
Hallar el determinante de la matriz $A=\begin{pmatrix} 
 2& 5& -3& -2 \\
 -2& -3& 2& -5 \\
 1& 3& -2& 2 \\
 -1& -6& 4& 3 \\
\end{pmatrix} $
haciendo ceros.

\tcblower
$det(A)=\begin{vmatrix} 
2 & 5 & -3 & -2 \\
-2 & -3 & 2 & -5 \\
1 & 3 & -2 & 2 \\
-1 & -6 & 4 & 3 \\ \end{vmatrix} 
{\color{red}\begin{array}{l} F_1  = F_1+2F_4 \\ F_2=F_2-2F_4 \\ F_3=F_3+F_4 \end{array}} =  \begin{vmatrix} 
0 & -7 & 5 & 4 \\
0 & 9 & -6 & -11 \\
0 & -3 & 2 & 5 \\
-1 & -6 & 4 & 3 \\ 
\end{vmatrix}  = 
(-1)(-1)^{4+1}\begin{vmatrix} 
-7 & 5 & 4 \\
9 & -6 & -11 \\
-3 & 2 & 5 \\ 
\end{vmatrix}  {\color{red}F_2=F_2+3F_3}  = 
\begin{vmatrix} 
-7 & 5 & 4 \\
0 & 0 & 4 \\
-3 & 2 & 5 \\ 
\end{vmatrix} = 4\cdot (-1)^{2+3} \begin{vmatrix} 
-7 & 5  \\
-3 & 2  \\ 
\end{vmatrix} = -4(-14+15)=-4
$
\end{ejemplo}




\section{Inversa de una matriz}

\begin{definicion}
La matriz inversa de una matriz cuadrada $A$ de orden $n$ es otra matriz $A^{-1}$ de orden $n$ que verifica:
\[
A\cdot A^{-1}=A^{-1} \cdot A = I
\]
Las matrices que tienen inversa se llaman \textbf{matrices regulares} y las que no tienen inversa se llaman \textbf{matrices singulares}.
\end{definicion}

Para calcular la matriz inversa vamos a utilizar tres procedimientos:
\begin{enumerate}
\item Mediante la definición.
\item Método de Gauss-Jordan.
\item Por determinantes.
\end{enumerate}

\subsection{Calculo mediante la definición}

En este procedimiento escribiremos cada elemento de la matriz inversa como una incógnita y mediante la definición de matriz inversa plantearemos un sistema de ecuaciones que luego resolveremos calculando de esta manera cada uno de los elementos de la matriz inversa.

\begin{ejemplo}
Dada la matriz    $A= \begin{pmatrix} 
 2& -1 \\
 -7& 4 \\
\end{pmatrix}$, hallar su inversa, si existe.

\tcblower
La matriz $A$ es de orden $2$, por lo tanto, la inversa también será de orden $2$.

Es decir, $A^{-1}=\begin{pmatrix} 
 x& y \\
 z& t \\ 
\end{pmatrix}$

Utilizaremos ahora la definición de matriz inversa y calcularemos las incógnitas y la matriz inversa.

$A\cdot A^{-1}= I \Leftrightarrow \begin{pmatrix} 
 2& -1 \\
 -7& 4 \\
\end{pmatrix} 
\begin{pmatrix} 
 x& y \\
 z& t \\ 
\end{pmatrix}
=\begin{pmatrix} 
 1& 0 \\
 0& 1 \\ 
\end{pmatrix}$

Operando e igualando las dos matrices:

$\begin{pmatrix} 
 2x-z& 2y-t \\
 -7x+4z& -7y+4t \\ 
\end{pmatrix}=\begin{pmatrix} 
 1& 0 \\
 0& 1 \\ 
\end{pmatrix} \Leftrightarrow
\left. 
\begin{array}{l}
 2x-z = 1 \\
 2y-t = 0 \\
 -7x+4z = 0 \\
 -7y+4t = 1 \\
\end{array}
\right\rbrace
\Rightarrow 
\begin{array}{l}
 x = 4 \\
 y =1 \\
 z =7 \\
 t = 2 \\
\end{array} $

Luego la matriz inversa de $A=\begin{pmatrix} 
 2& -1 \\
 -7& 4 \\
\end{pmatrix}$ es:
$A^{-1}= \begin{pmatrix}
 4& 1 \\
 7& 2 \\
\end{pmatrix}$
\end{ejemplo}

\subsection{Calculo de la matriz inversa por el método de Gauss-Jordan}

En este método calcularemos la matriz inversa de una matriz mediante transformaciones elementales.

Este método consiste en lo siguiente:

\begin{itemize}
\item Tomamos una matriz formada por la matriz $A$ y la matriz identidad del mismo orden. Esta matriz la simbolizaremos por $\left( \begin{array}{c | c}
A &  I 
\end{array} \right) $
\item Realizamos transformaciones elementales para llegar a la matriz: $\left( \begin{array}{c | c} I & B 
\end{array} \right) $
\end{itemize}

La matriz $B$ es la inversa de la matriz $A$, es decir $A^{-1}$


Las transformaciones elementales son las siguientes:

\begin{enumerate}
\item Intercambiar dos filas. $F_i \leftrightarrow F_j$
\item Multiplicar una fila por un número distinto de cero. $F_i \rightarrow kF_i$
\item Sumar dos filas multiplicadas por sendos números y sustituir una de estas filas por el resultado. $F_i \rightarrow kF_i+tF_j$.
\end{enumerate}

\begin{ejemplo}
 Hallar la inversa de  la matriz:$\begin{pmatrix}
1 & 2  \\
2 & 3   \end{pmatrix}$.

\tcblower

$
\begin{gaussjordandos}
1 & 2  & 1& 0 \\
  2& 3  & 0& 1 \\
\end{gaussjordandos}
\begin{array}{l}
F_2=F_2-2F_1 \\
F_2=F_2-2F_1
\end{array}
$

$
\begin{gaussjordandos}
 1& 2  & 1& 0 \\
 0& -1  & -2& 1 \\
\end{gaussjordandos}
\begin{array}{l}
F_1=F_1+2F_1 \\
\\
\end{array}
\begin{gaussjordandos}
 1& 0  & -3& 2 \\
 0& -1  & -2& 1 \\
\end{gaussjordandos}
\begin{array}{l}
\\
F_2=-F_2 
\end{array}
$

$
\begin{gaussjordandos}
 1& 0 &  -3& 2 \\
 0& 1  & 2& -1 \\
\end{gaussjordandos}
$

Luego la matriz inversa de $A$ es $A^{-1}=\begin{pmatrix}
  -3& 2 \\
  2& -1 \\
\end{pmatrix}$

\end{ejemplo}

\begin{ejemplo}
Hallar la inversa de la matriz $B=\begin{pmatrix}
1 & -1 & 1 \\
2 & 0 & 1 \\
-1 & 1 & 1
\end{pmatrix}$

\tcblower
$ \begin{gaussjordantres} 
1 & -1 & 1 & 1 & 0 &0 \\
2 & 0 & 1 & 0 & 1 & 0 \\
-1 & 1 & 1 & 0 & 0 & 1
\end{gaussjordantres} $

$\begin{array}{l}  \\  \rightarrow f_2=f_2-2f_1 \\  \rightarrow f_3=f_3+f_1 \end{array}$

$\begin{gaussjordantres} 
1 & -1 & 1 & 1 & 0 &0 \\
0 & 2 & -1 & -2 & 1 & 0 \\
0 & 0 & 2 & 1 & 0 & 1
\end{gaussjordantres}$
$\begin{array}{l} \rightarrow f_1=2f_1+f_2 \\ \\ \\ \end{array}$

$\begin{gaussjordantres} 
2 & 0 & 1 & 0 & 1 &0 \\
0 & 2 & -1 & -2 & 1 & 0 \\
0 & 0 & 2 & 1 & 0 & 1
\end{gaussjordantres} $

$\begin{array}{l} \rightarrow f_1=2f_1-f_3 \\ \rightarrow f_2=2f_2+f_3 \\ \\ \end{array} $

$\begin{gaussjordantres} 
4 & 0 & 0 & -1 & 2 & -1 \\
0 & 4 & 0 & -3 & 2 & 1 \\
0 & 0 & 2 & 1 & 0 & 1
\end{gaussjordantres}$

$\begin{array}{l} \rightarrow f_1=f_1/4 \\ \rightarrow f_2=f_2/4 \\ \rightarrow f_3=f_3/2 \end{array}  $

$\begin{gaussjordantres} 
1 & 0 & 0 & -1/4 & 1/2 & -1/4 \\
0 & 1 & 0 & -3/4 & 1/2 & 1/4 \\
0 & 0 & 1 & 1/2 & 0 & 1/2
\end{gaussjordantres} $

$B^{-1}=\begin{pmatrix}
-1/4 & 1/2 & -1/4 \\
 -3/4 & 1/2 & 1/4 \\
 1/2 & 0 & 1/2
\end{pmatrix} $

\end{ejemplo}
\subsection{Cálculo de la matriz inversa por determinantes}

\begin{definicion}
La condición necesaria y suficiente para que una matriz tenga inversa es que su determinante sea distinto de cero.
\[ \exists A^{-1} \Leftrightarrow |A| \neq 0 \]

Si existe la matriz inversa se puede calcular de la siguiente forma:


\resaltado{
$ A^{-1}= \dfrac{1}{|A|}\cdot \left[ Adj (A) \right]^t = \dfrac{1}{|A|}\cdot Adj  \left( A^t \right) $} 

\end{definicion}


\begin{ejemplo}
Hallar la matriz inversa de :
$A=\begin{pmatrix}
1 & 0 & 0 \\
0 & 0 & -1 \\
2 & -1 & 1
\end{pmatrix} $

\tcblower
Hallamos el determinante de $A$. $|A|=0+0+0-0-1-0=-1 $

Como el determinante es distinto de cero la matriz tiene inversa.

Hallamos la matriz adjunta de $A$.

$\begin{matrix}
A_ {11}=\begin{vmatrix}
0 & -1 \\
-1 & 1
\end{vmatrix}=-1 &  A_ {12}=-\begin{vmatrix}
0 & -1 \\
2 & 1
\end{vmatrix}=-2 & A_ {13}=\begin{vmatrix}
0 & 0 \\
2 & -1
\end{vmatrix}=0 \\
A_ {21}=-\begin{vmatrix}
0 & 0 \\
-1 & 1
\end{vmatrix}=0 & A_ {22}=\begin{vmatrix}
1 & 0 \\
2 & 1
\end{vmatrix}=1 & A_ {23}=-\begin{vmatrix}
1 & 0 \\
2 & -1
\end{vmatrix}=1 \\
A_ {31}=\begin{vmatrix}
0 & 0 \\
0 & -1
\end{vmatrix}=0 & A_ {32}=-\begin{vmatrix}
1 & 0 \\
0 & -1
\end{vmatrix}=1 & A_ {33}=\begin{vmatrix}
1 & 0 \\
0 & 0
\end{vmatrix}=0
\end{matrix} $

$(Adj A)= \begin{pmatrix}
-1 & -2 & 0 \\
0 & 1 & 1 \\
0 & 1 & 0
\end{pmatrix} \qquad \left( Adj A \right)^t=\begin{pmatrix}
-1 & 0 & 0 \\
-2 & 1 & 1 \\
0 & 1 & 0
\end{pmatrix}  $

\resultado{
$A^{-1}= \dfrac{\left( Adj A \right)^t}{|A|}=\begin{pmatrix}
1 & 0 & 0 \\
2 & -1 & -1 \\
0 & -1 & 0
\end{pmatrix}  $ 
}
\end{ejemplo}

\begin{ejemplo}
Sea $m$ un número real y considerese la matriz 
\[ A=\begin{pmatrix}
1 & 0 & m \\
m & 0 & -1 \\
2 & -1 & 1
\end{pmatrix} \]
\begin{enumerate}[label=\alph* )]
\item Determine todos los valores de $m$ para los que la matriz $A$ tiene inversa.
\item Determine, si existe, la inversa de $A$ cuando $m=0$
\item Determine, si existe, la inversa de $A^2$ cuando $m=0$
\end{enumerate}

\tcblower

\begin{enumerate}[label=\alph* )]
\item $\exists A^{-1} $ si $|A| \neq 0$

$|A|=\begin{vmatrix}
1 & 0 & m \\
m & 0 & -1 \\
2 & -1 & 1
\end{vmatrix}=-m^2-1$

$-m^2-1 =0 \Rightarrow $ No tiene solución, por lo tanto, $|A| \neq 0$ para $\forall m \in \R$

Luego la matriz siempre tiene inversa.
\item Para $m=0 \qquad A=\begin{pmatrix}
1 & 0 & 0 \\
0 & 0 & -1 \\
2 & -1 & 1
\end{pmatrix} $

$|A|=-1 $

Hallamos la matriz adjunta de $A$.

$\begin{matrix}
A_ {11}=\begin{vmatrix}
0 & -1 \\
-1 & 1
\end{vmatrix}=-1 & A_ {12}=-\begin{vmatrix}
0 & -1 \\
2 & 1
0\end{vmatrix}=-2 & A_ {13}=\begin{vmatrix}
0 & 0 \\
2 & -1
\end{vmatrix}=0 \\
A_ {21}=-\begin{vmatrix}
0 & 0 \\
-1 & 1
\end{vmatrix}=0 & A_ {22}=\begin{vmatrix}
1 & 0 \\
2 & 1
\end{vmatrix}=1 & A_ {23}=-\begin{vmatrix}
1 & 0 \\
2 & -1
\end{vmatrix}=1 \\
A_ {31}=\begin{vmatrix}
0 & 0 \\
0 & -1
\end{vmatrix}=0 & A_ {32}=-\begin{vmatrix}
1 & 0 \\
0 & -1
\end{vmatrix}=1 & A_ {33}=\begin{vmatrix}
1 & 0 \\
0 & 0
\end{vmatrix}=0
\end{matrix} $

$(Adj A)= \begin{pmatrix}
-1 & -2 & 0 \\
0 & 1 & 1 \\
0 & 1 & 0
\end{pmatrix} $

$\left( Adj A \right)^t=\begin{pmatrix}
-1 & 0 & 0 \\
-2 & 1 & 1 \\
0 & 1 & 0
\end{pmatrix}  $

$A^{-1}= \dfrac{\left( Adj A \right)^t}{|A|}=\begin{pmatrix}
1 & 0 & 0 \\
2 & -1 & -1 \\
0 & -1 & 0
\end{pmatrix}  $
\item $\left( A^2 \right)^{-1}= (A \cdot A)^{-1}=A^{-1} \cdot A^{-1}=\begin{pmatrix}
1 & 0 & 0 \\
2 & -1 & -1 \\
0 & -1 & 0
\end{pmatrix} \cdot \begin{pmatrix}
1 & 0 & 0 \\
2 & -1 & -1 \\
0 & -1 & 0
\end{pmatrix} = \begin{pmatrix}
1 & 0 & 0 \\
0 & 2 & 1 \\
-2 & 1 & 1
\end{pmatrix} $
\end{enumerate}

\end{ejemplo}


\begin{ejer}
\bex
\itemps {Discute, en función del parámetro $m \in \R $, el rango de la matriz:
\[ \begin{pmatrix}
1 & 3 & -1 \\
m+1 & 3 & m-1 \\
m-1 & m+3 & -1
\end{pmatrix} \] }{$\begin{array}{l} m \in \R - \left\lbrace 0,1 \right\rbrace \Rightarrow  \text{rango}(A)=3 \\
m=0 \Rightarrow  \text{rango}(A)=2 \\
m=1 \Rightarrow  \text{rango}(A)=2. \end{array}$}

\itemps {¿Para qué valores del parámetro $m \in \R $ existe la matriz inversa de $A$?}{$m \in \R - \left\lbrace 0,1 \right\rbrace  $}
\eex
\end{ejer}

\begin{ejer}
Considerese la matriz $M=\begin{pmatrix}
1 & a & a^2 \\
1 & a+1 & (a+1)^2 \\
1 & a-1 & (a-1)^2
\end{pmatrix} $, para $a \in \R$.
\bex
\itemps {Calcule el rango de la matriz $M$ en función de los valores del parámetro $a$.}{}
\itemps {Discuta y resuelva el sistema de ecuaciones lineales
\[ M \cdot \begin{pmatrix}
x \\
y \\
z 
\end{pmatrix} = \begin{pmatrix}
1 \\
1 \\
1
\end{pmatrix} \]
según los valores del parámetro $a$.}{}
\eex
\end{ejer}

\section{Rango de una matriz}

\begin{definicion}
Una fila (o columna) de una matriz $F_i$ (o $C_i$) depende linealmente de otras si existen números reales $a_1$, $a_2$, $\cdots$, $a_n$ tales que $F_i$ (o $C_i$) se pueden expresar como:
\[ F_i=a_1 \cdot F_1+a_2 \cdot F_2+\cdots + a_n \cdot F_n \qquad C_i=a_1\cdot C_1+a_2 \cdot C_2+\cdots + a_n \cdot C_n \]
\end{definicion}

\begin{definicion}
El \textbf{rango de una matriz} es el número de filas o de columnas linelamente independientes.
\end{definicion}

Se puede demostrar que el número de filas linealmente independientes coincide con el número de columnas linalmente independiente. Por lo tanto, el rango de una matriz de orden $m \times n$ siempre será menor o igual que el número más pequeño entre $m$ y $n$.

\begin{ejemplo}
Dada la matriz $A=\begin{pmatrix}
-1 & 2 & 4 & 0 \\
3 & 2 & 1 & -2 \\
7 & 2 & -2 & -4 
\end{pmatrix}$.

Comprueba que la tercera fila es combinación lineal de las dos primeras. ¿Cuál es el rango de $A$?.
\tcblower
Si la tercera fila es combinación lineal de la primera entonces: $F_3=aF_1+bF_1$.

Aplicando la igualdad anterior tenemos:

$\begin{cases} -a+3b=7 \\ 2a+2b=2 \\ 4a+b=-2 \\ -2b=-4 \end{cases}$

Resolviendo el sistema obtenemos $a=-1$, $b=2$. Es decir, $F_3=-F_1+2F_2$

Como la primera fila y la segunda son independientes el rango de la matriz $A$ será $2$.
\end{ejemplo}

\subsection{Calculo del rango de una matriz escalonando la matriz}
\begin{definicion}
Una matriz se dice que es una \textbf{matriz escalonada} si en ella se cumple:
\begin{itemize}
\item Si hay filas nulas, están situadas en la parte inferior de la matriz.
\item En las filas no nulas, el primer elemento diferente de cero de una fila está situado más a la derecha que el primer elemento no nulo de la fila inmediatamente superior.
\end{itemize}

El rango de una matriz escalonada es el número de filas no nulas de ella. Lo denotaremos por $ran(A)$.
\end{definicion}

Las siguientes matrices son matrices escalonadas:

$A=\begin{pmatrix}
3 & -1 & 1 \\ 0 & 2 & 5 
\end{pmatrix}
$, $B=\begin{pmatrix}
1 & 2 & 3 & 4 \\
0 & 3 & -2 & 1 \\
0 & 0 & -1 & 2 \\
0 & 0 & 0 & 1 \\
0 & 0 & 0 & 0 
\end{pmatrix}$


\begin{definicion}
Dos matrices son equivalentes si una de ellas se obtiene a partir de la otra mediante transformaciones elementales.

El rango de una matriz A es el rango de una matriz escalonada equivalente a ella.
\end{definicion}

\begin{ejemplo}

Calcula el rango de la matriz:
$ 
\begin{pmatrix}
	1 & 2 & 5 & 6 \\
	-1 & -2 & 0 & 1 \\
	3 & 4 & 5 & 3 \\
	-3 & -4 & 0 & 0  
\end{pmatrix}
$

\tcblower

Escalonamos la matriz haciendo ceros en todos los elementos por debajo de la diagonal:

\[
\begin{pmatrix}
	1 & 2 & 5 & 6 \\
	-1 & -2 & 0 & 1 \\
	3 & 4 & 5 & 3 \\
	-3 & -4 & 0 & 0  
\end{pmatrix} 
\xLongrightarrow{\begin{array}{c} f_2=f_2+f_1 \\ f_3=f_3-3f_1 \\  f_4=f_4+3f_1 \end{array}}
\begin{pmatrix}
	1 & 2 & 5 & 6 \\
	-0 & 0 & 5 & 7 \\
	0 & -2 & -10 & -15 \\
	-0 & 2 & 15 & 18  
\end{pmatrix} 
\xLongrightarrow{\begin{array}{c} f_2\leftrightarrow f_3  \end{array}}
\]
\[
\begin{pmatrix}
	1 & 2 & 5 & 6 \\
	0 & -2 & -10 & -15 \\
	0 & 0 & 5 & 7 \\
	0 & 2 & 15 & 18  
\end{pmatrix} 
\xLongrightarrow{\begin{array}{c} f_4=f_4+f_2  \end{array}}
\begin{pmatrix}
	1 & 2 & 5 & 6 \\
	0 & -2 & -10 & -15 \\
	0 & 0 & 5 & 7 \\
	0 & 0 & 5 & 3 
\end{pmatrix} 
\xLongrightarrow{\begin{array}{c} f_4=f_4-f_3  \end{array}}
\]
\[
\begin{pmatrix}
	1 & 2 & 5 & 6 \\
	0 & -2 & -10 & -15 \\
	0 & 0 & 5 & 7 \\
	0 & 0 & 0 & -4  
\end{pmatrix} 
\]

Nos quedan cuatro filas, por lo tanto, el rango es 4.

\end{ejemplo}
\subsection{Calculo del rango de una matriz por determinantes}

\begin{definicion}
Se llama menor de orden $k$ de una matriz $A$, de cualquier dimensión, al determinante de la matriz formada por elementos que perteneces a $k$ filas y $k$ columnas de la matriz A.
\end{definicion}

\begin{ejemplo}
Hallar un menor de orden 2 y otro de orden 3 de la matriz: $A=\begin{pmatrix}
1 & 0 & 2 & -3 \\
3 & 3 & 1 & 2 \\
0 & -1 & 2 & -3
\end{pmatrix}$
\tcblower
Para hallar un menor de orden 2 de esta matriz elegimos dos filas y dos columnas de dicha matriz. Por ejemplo, elegimos la fila 2 y la fila 3 y la columna 1 y la columna 4 y obtenemos:

$\begin{vmatrix}
3 & 2 \\
0 & -3
\end{vmatrix}= -9$

Para hallar un menor de orden 3 cogemos tres filas, por ejemplo, las filas 1, 2 y 3 y tres columnas, por ejemplo las columnas 1,2 y 4. De esta manera obtenemso:

$\begin{vmatrix}
1 & 0 & -3 \\
3 & 3 & 2 \\
0 & -1 & -3
\end{vmatrix}= 2$
\end{ejemplo}

\begin{definicion}
El rango de una matriz es el orden del mayor menor no nulo de la matriz.
\end{definicion}

Para calcular el rango de una matriz mediante determinantes procederemos de la siguiente forma:

\begin{itemize}
\item Si la matriz $A$ es la matriz nula el rango será 0. En caso contrario buscamos un menor de orden 2 que sea distinto de cero. Si no hay ninguno el rango será 1.
\item Si hay algún menor de orden 2 distinto de cero el rango será mayor o igual a 2. En este caso orlamos el menor de orden 2 distinto de cero añadiendole otra fila y otra columna utilizando todas las posibilidades. Si todos son cero el rango será 2 en caso contrario el rango será mayor o igual que 3 y se sigue orlando el menor de orden 3 distinto de cero si esto es posible, es decir, si hay filas y columnas para añadirle.
\end{itemize}
El rango de la matriz será el orden del mayor menor no nulo.

\begin{ejemplo}
Hallar el rango de la matriz $A=\begin{pmatrix}
-1 & 0 & 0 & 2 & -2 \\
3 & 1 & -1 & 4 & 0 \\
2 & 1 & -1 & 6 & -2 
\end{pmatrix}$
\tcblower
Buscamos un menor de orden 2 no nulo. Para ello tomamos la primera y la segunda fila y la primera y la segunda columna.

$\begin{vmatrix}
-1 & 0 \\ 3 & 1
\end{vmatrix}=-1 \neq 0 \Rightarrow \text{ran}(A) \geq 2 $

Orlamos este menor añadiéndole otra fila y otra columna hasta encontrar un menor de orden 3 distinto de cero si lo hay.

$ \begin{vNiceMatrix}[first-row,first-col]
 &  & & \color{red}\underset{\downarrow}{C_3} \\
 & -1 & 0 & 0 \\
 & 3 & 1 & -1 \\
\color{red} F_3 \rightarrow  & 2 & 1 & -1 
\end{vNiceMatrix}=0 \qquad  
\begin{vNiceMatrix}[first-row,first-col]
 &  & & \color{red}\underset{\downarrow}{C_4} \\
 & -1 & 0 & 2 \\
 & 3 & 1 & 4 \\
\color{red} F_3 \rightarrow  & 2 & 1 & 6
\end{vNiceMatrix}=0 \qquad  
\begin{vNiceMatrix}[first-row,first-col]
 &  & & \color{red}\underset{\downarrow}{C_5} \\
 & -1 & 0 & -2 \\
 & 3 & 1 & 0 \\
\color{red} F_3 \rightarrow  & 2 & 1 & -2
\end{vNiceMatrix}=0$

Por lo tanto, el rango de la matriz es $2$.
\end{ejemplo}

\begin{ejer}
Utilizando el método de Gauss, halla el rango de la siguiente matriz:
\[
\begin{pmatrix}
1 & 2& 5 & 6 \\
-1 & -2 & 0 & 1 \\
3 & 4 &5 & 3 \\
-3 & -4 & 0 & 0
\end{pmatrix} \]
\begin{solu}
ran$(A)=4$
\end{solu}
\end{ejer}

\begin{ejer}
 Calcula el rango de la matriz $\begin{pmatrix}
-1 & -1 & -1 \\
3 & 6 & 9 \\
-5 & -10 & m
\end{pmatrix}$ 
según los valores del parámetro $m$.
\begin{solu}
$m \neq -15 \rightarrow $ rango = $3$ \\ $m = -15 \rightarrow $ rango= $2$
\end{solu}
\end{ejer}


\section{Ecuaciones matriciales}

\begin{ejemplo}
Resolver la siguiente ecuación matricial $X \cdot A=B-C$, siendo:

$A=\begin{pmatrix}
5 & 2 \\
3 & 1
\end{pmatrix}$, $B=\begin{pmatrix}
2 & 1 \\
3 & -2
\end{pmatrix}$ y $C=\begin{pmatrix}
1 & -1 \\
1 & 2
\end{pmatrix}$

\tcblower
Primero despejamos la matriz incógnita $X$. Para ello utilizaremos la matriz inversa.

$X \cdot A=B-C \Rightarrow  X \cdot A\cdot A^{-1}=(B-C)\cdot A^{-1} \Rightarrow X \cdot I=(B-C)\cdot A^{-1} \Rightarrow X=(B-C)\cdot A^{-1}$

Calculamos la inversa de la matriz $A$ y luego realizamos las operaciones.

$|A|=\begin{vmatrix}
5 & 2 \\
3 & 1
\end{vmatrix}= -1$

$\begin{matrix}
A_{11}=|1|= 1 & A_{12}=-|3|=-3 \\
A_{21}=-|2|= -2 & A_{22}=|5|=5 
\end{matrix} $

$adj (A)= \begin{pmatrix}
1 & -3 \\
-2 & 5
\end{pmatrix} \Rightarrow \left( adj (A) \right)^t=\begin{pmatrix}
1 & -2 \\
-3 & 5
\end{pmatrix}$

$A^{-1}=\begin{pmatrix}
-1 & 2 \\
3 & -5
\end{pmatrix}$

$X=\left[ \begin{pmatrix}
2 & 1 \\
3 & -2
\end{pmatrix} -\begin{pmatrix}
1 & -1 \\
1 & 2
\end{pmatrix} \right]\begin{pmatrix}
-1 & 2 \\
3 & -5
\end{pmatrix} = \begin{pmatrix}
1 & 2 \\
2 & -4
\end{pmatrix} \begin{pmatrix}
-1 & 2 \\
3 & -5
\end{pmatrix} = \begin{pmatrix}
5 & -8 \\
-14 & 24
\end{pmatrix} $
\end{ejemplo}

\begin{ejemplo}
Determina una matriz $X$ tal que $A+2\cdot X \cdot B=C$, siendo:
\[ A=\begin{pmatrix}
1 & -2 & 1 \\
0 & 3 & 1 
\end{pmatrix}
\qquad
B=\begin{pmatrix}
1 & -1 & 1 \\
2 & 0 & 1 \\
-1 & 1 & 1
\end{pmatrix}
\qquad
C=\begin{pmatrix}
1 & 2 & 3 \\ 
8 & -1 & -1 
\end{pmatrix} \]

\tcblower
Primero despejamos la matriz $X$.

\[ A+2\cdot X \cdot B=C \rightarrow 2\cdot X \cdot B=C - A \rightarrow  X \cdot B=\left(\dfrac{C-A}{2} \right) \rightarrow \]
\[ \rightarrow X\cdot B \cdot B^{-1}=\left(\dfrac{C-A}{2} \right) \cdot B^{-1} \rightarrow X=\left(\dfrac{C-A}{2} \right) \cdot B^{-1} \]

Calculamos la inversa de $B$

\[ \begin{matrizgauss} 
1 & -1 & 1 & 1 & 0 &0 \\
2 & 0 & 1 & 0 & 1 & 0 \\
-1 & 1 & 1 & 0 & 0 & 1
\end{matrizgauss}
\begin{array}{c}  \\ f_2=f_2-2f_1 \\ f_3=f_3+f_1 \end{array}
\begin{matrizgauss} 
1 & -1 & 1 & 1 & 0 &0 \\
0 & 2 & -1 & -2 & 1 & 0 \\
0 & 0 & 2 & 1 & 0 & 1
\end{matrizgauss}
\begin{array}{c} f_1=2f_1+f_2 \\ \\ \\ \end{array} \]
\[ \begin{matrizgauss} 
2 & 0 & 1 & 0 & 1 &0 \\
0 & 2 & -1 & -2 & 1 & 0 \\
0 & 0 & 2 & 1 & 0 & 1
\end{matrizgauss} 
\begin{array}{c} f_1=2f_1-f_3 \\ f_2=2f_2+f_3 \\ \\ \end{array} 
\begin{matrizgauss} 
4 & 0 & 0 & -1 & 2 & -1 \\
0 & 4 & 0 & -3 & 2 & 1 \\
0 & 0 & 2 & 1 & 0 & 1
\end{matrizgauss}
\begin{array}{c} f_1=f_1/4 \\ f_2=f_2/4 \\ f_3=f_3/2 \end{array}  
 \]
\[ \begin{matrizgauss} 
1 & 0 & 0 & -1/4 & 1/2 & -1/4 \\
0 & 4 & 0 & -3/4 & 1/2 & 1/4 \\
0 & 0 & 2 & 1/2 & 0 & 1/2
\end{matrizgauss} \]

Luego $B^{-1}=\begin{pmatrix}
-1/4 & 1/2 & -1/4 \\
 -3/4 & 1/2 & 1/4 \\
 1/2 & 0 & 1/2
\end{pmatrix} $

Ahora calculamos $\left( \dfrac{C-A}{2} \right) = 
\dfrac
{\begin{pmatrix}
1 & 2 & 3 \\ 
8 & -1 & -1 
\end{pmatrix} - 
\begin{pmatrix}
1 & -2 & 1 \\
0 & 3 & 1 
\end{pmatrix}}{2}= 
\begin{pmatrix}
0 & 2 & 1 \\
4 & -2 & -1 
\end{pmatrix}$

Por tanto: $X=\begin{pmatrix}
0 & 2 & 1 \\
4 & -2 & -1 
\end{pmatrix} 
\cdot
\begin{pmatrix}
-1/4 & 1/2 & -1/4 \\
 -3/4 & 1/2 & 1/4 \\
 1/2 & 0 & 1/2
\end{pmatrix}=
\begin{pmatrix}
-1 & 1 & 1 \\
0 & 1 & -2 
\end{pmatrix}$
\end{ejemplo}

\begin{ejer}
Dadas las matrices 
$A=\begin{pmatrix}
3 & 2 \\
2 & 4
\end{pmatrix} $ y 
$B=\begin{pmatrix}
2 & 5 \\
-3 & 1
\end{pmatrix}$, resuelve la ecuación matricial $A\cdot X+B^t=B$.
\begin{solu}
\end{solu}
\end{ejer}

\begin{ejer}
 Sean las matrices :
\[ A=\begin{pmatrix}
-1 & 4 & -1 \\
0 & -1 & 0 \\
3 & 1 & 2 
\end{pmatrix}
\qquad
B=\begin{pmatrix}
-2 & 1 & 3 \\
0 & 2 & -1 \\
1 & 0 & 1 
\end{pmatrix}
\qquad
C=\begin{pmatrix}
5 & -2 & -6 \\
0 & -3 & 2 \\
-2 & 0 & -1 
\end{pmatrix} \]
Determine $X$ en la ecuación matricial $X \cdot A -2B=C$
\begin{solu}
\end{solu}
\end{ejer}

\begin{ejer}
 Sean 
$A=\begin{pmatrix}
 1 & 1 & 1 \\
 1 & 2 & 5 \\
 1 & 4 & 2
\end{pmatrix} $ y 
$B=\begin{pmatrix}
 1 & 1 & 0 \\
 1 & 0 & 1 \\
 0 & 1 & 1 
\end{pmatrix} $
\bex
\itemps {Calcule $B^{-1}$}{}
\itemps {Utilizando $B^{-1}$, calcule $X$ tal que $XB=A+B$}{}
\eex 
\begin{solu}
\end{solu}
\end{ejer}

\begin{ejer}
 Dada la ecuación matricial $I+A \cdot X-A^2\cdot X=B$. Se pide:
\bex
\itemps {Resuelve matricialmente la ecuación.}{}
\itemps {Si $A=\begin{pmatrix}
 1 & 2 \\
 -1 & 1
\end{pmatrix} $, calcula la matriz $A-A^2$}{}
\itemps {Siendo $A$ la matriz anterior, $B=\begin{pmatrix}
 -3 & -4 \\
 7 & 11
\end{pmatrix} $ e $I=\begin{pmatrix}
 1 & 0 \\
 0 & 1
\end{pmatrix} $ calcula la matriz $X$.}{}
\eex
\end{ejer}


\begin{ejer}
Determinar la matriz $X$ solución de la ecuación matricial $A\cdot X -A \cdot B= B \cdot X$ donde:
\[ A=\begin{pmatrix}
 2 & -1 \\
 1 & 0 
\end{pmatrix}
\qquad
B=\begin{pmatrix}
 1 & -1 \\
 -2 & 1 
\end{pmatrix} \]
\begin{solu}
\end{solu}
\end{ejer}


\newpage

\section{Problemas con matrices}

%\tituloejercicios



\begin{ejer} 
Tres escritores presentan a un editor, al acabar una enciclopedia, la minuta siguiente:

\begin{center}
\begin{tabular}{|c|c|c|c|} 
\hline
&Horas de trabajo & Conferencias dadas & Viajes \\
\hline
Escritor A & 40 & 10 & 5 \\ 
\hline
Escritor B & 80 & 15 & 8 \\
\hline
Escritor C & 100 & 25 & 10 
\\
\hline
\end{tabular}
\end{center}

El editor paga la hora de trabajo a 75 euros, la conferencia a 30 euros y el viaje a 50 euros. Si solo piensa pagar, respectivamente, el 30\%, el 20\% y el 10 \% de lo que le corresponde a cada escritor, ¿qué gasto tendría el editor?. Resuelvelo con matrices.
\begin{solu}
\end{solu}
\end{ejer}

\begin{ejer} 
 Una empresa de carpintería dispone de dos naves A y B donde se fabrican sillas y mesas de tres tipos de acabados: calidad extra E, calidad media M y calidad inferior I. Ambas naves tienen la misma producción mensual. La cantidad de sillas producidas mensualmente, en cada una de las naves, es de 100 del tipo E, 150 del M y 200 del tipo I; la producción mensual de mesas es de 100 de clase E, 50 de clase M y 300 de clase I. Se sabe que el porcentaje de sillas y mesas defectuosas es, en la nave A, de $0.01$ para los muebles de calidad E, de $0.02$ para los de calidad M y de $0.03$ para los de calidad I, mientras que en la nave B, los porcentajes son $0.02$ para la calidad E, $0.04$ para la calidad M y $0.01$ para la calidad I. Se pide:
\bex
\itemps {Obtener la matriz que representa la producción de sillas y mesas, de calidad extra, media e inferior en cada una de las naves.}{}
\itemps {Obtener la matriz que representa el número de sillas y mesas defectuosas, en las tres calidades, procedentes de cada una de las naves y la matriz que da el número total de sillas y mesas defectuosas para cada calidad.}{}
\eex

\end{ejer}

\begin{ejer} 
En un colegio se imparten los cursos de 1º, 2º y 3º. Los profesores tienen asignado un número de horas de clase, tutorías y guardias de acuerdo con la siguiente matriz:

$M=\bordermatrix{
& C & G & T \cr
1\g & 20 & 5 & 3 \cr
2\g & 18 & 6 & 5 \cr
3\g & 22 & 1 & 2 } $

El colegio paga cada hora clase a 20 euros, cada hora de guardia a 5 euros y cada hora de tutoría a 10 euros, según la matriz: $C=\begin{pmatrix}
20 \\
5 \\
10
\end{pmatrix} $.

El colegio dispone de 5 profesores para primer curso, 4 para segundo y 6 para tercero representados por la matriz $P=\begin{pmatrix}51-52
5 & 4 & 6
\end{pmatrix}$.

Calcula cada uno de los siguientes productos de matrices e interpreta los resultados:

\bex
\itemps {$P\cdot M$}{}
\item {$M \cdot C$}{}
\item {$P \cdot M \cdot C$}{}
\eex
\end{ejer}

\begin{ejer} 
 Una cadena de hoteles posee tres hoteles en una ciudad: Excelsior, Palace y Oasis. Cada hotel dispone de tres tipos de habitaciones: de lujo, doble e individual. El Excelsior posee 6 habitaciones de lujo, 30 dobles y 10 individuales. El Palace 4, 50 y 10 respectivamente y el Oasis 4, 50 y 8. El precio por habitación y noche es 200 euros la de lujo, 100 la doble y 75 la individual.

\bex
\itemps {Recoge estos datos en dos matrices indicando qué significa cada fila y columna.}{}
\itemps {Suponiendo que estuvieran completos una noche, expresar mediante una matriz los ingresos obtenidos por cada hotel.}{}
\eex
\end{ejer}

\begin{ejer}  En un centro educativo los datos de matrícula por cursos son: en 1º 50 alumnos y 70 alumnas, en 2º 60 alumnos y 50 alumnas, en 3º 45 alumnos y 60 alumnas y en 4º 50 alumnos y 55 alumnas.

Un estudio revela que cada alumno/a de 3º lee cada año 2 novelas, ningún libro de poesía y un libro de otros temas. En 2º 3 novelas, 1 libro de poesía y 2 de otros temas. En 3º 4 novelas, 2 libros de poesía y 1 de otros temas. En 4º 2 novelas, 1 libro de poesía y ninguno de otros temas.
\bex
\itemps {Disponer la información acerca de la matrícula y lectura en dos matrices.}{}
\itemps {Explicar qué representan cada uno de los elementos de la matriz producto de las dos matrices anteriores.}{}
\itemps {¿Cuántos libros de novela leen por curso todas las alumnas matriculadas en el centro?}{}
\eex

\end{ejer}


\section{Problemas}

\begin{ejer}
Sea $A$ una matriz de orden 3 con sus elementos en los números reales tal que $A^{-1}=A^t$
\bex
\itemps {Obtenga los posibles valores del determinante de $A$}{$\pm 1 $}
\itemps {Halle los posibles valores del determinante de $A^n$, cuando $n$ es un número natural}{$\pm 1$}
\itemps {Determine los posibles valores del determinante de $A^{-1}$}{$\pm 1$}
\eex
\end{ejer}

\begin{ejer}
Dado el número real $a$ considere la matriz $A=\matriztres{1}{-1}{-a}{a}{3}{1}{-2}{a}{2a} $
\bex
\itemps {Obtenga los valores del número real $a$ para los cuales la matriz $A$ tiene inversa}{•}
\itemps {Calcule, si es posible, la inversa de $A$ cuando $a=0$}{•}
\eex
\end{ejer}

\begin{ejer}
Dados los números reales $a$, $b$, $c$ y $x$ se considera la matriz $A=\matriztres{a}{b}{c}{a}{x}{c}{a}{b}{x} $
\bex
\itemps {Halle los valores de $x$ para los cuales el determinante de $A$ es nulo para cualesquiera valores de $a$, $b$ y $c$}{$x=b$, $x=c$}
\itemps {Si $x=1$, $b=c=2$, halle los valores de $a$ para los cuales $A$ tiene inversa}{$a \neq 0$}
\itemps {Halle, si es posible, la inversa de $A$ cuando $x=0$ y $a=b=c=1$}{$A^{-1}=\matriztres{-1}{1}{1}{1}{-1}{0}{1}{0}{-1}$}
\eex
\end{ejer}

\begin{ejer}
Determine la relación que debe existir entre los parámetros $x$ e $y$ para que las matrices $A=\begin{pmatrix}
x & 1 \\ 1  & y 
\end{pmatrix}$ y $B=\begin{pmatrix}
1 & x \\ y & 1
\end{pmatrix}$ conmuten, es decir, para que $A\cdot B=B\cdot A$		
\begin{solu}
$x=y$
\end{solu}
\end{ejer}

\begin{ejer}
Hallar $X$ e $Y$, matrices $2\times 2$, tales que:
\[ \begin{cases}  X+\begin{pmatrix} 3 & -1 \\ 0 & 2 \end{pmatrix} Y= \begin{pmatrix} 2 & 1 \\ 1 & 3 \end{pmatrix} \\
X+\begin{pmatrix} 1 & 0 \\ 1 & 1 \end{pmatrix} Y= \begin{pmatrix} 1 & 3 \\ 0 & 1 \end{pmatrix} \end{cases} \]
\begin{solu}

\end{solu}
\end{ejer}
\begin{ejer}
Considere las siguientes matrices:
\[ A=\begin{pmatrix}
1 & 1 \\
-1 & -1
\end{pmatrix}	\qquad B=\begin{pmatrix}
0 \\ 1
\end{pmatrix} \]


\bex
\itemps {Calcular $C=A^t \cdot A - B \cdot B^t$, donde $A^t$ y $B^t$ denotan, respectivamente, las matrices traspuestas de $A$ y $B$.}{$C=\begin{pmatrix}
2 & 2 \\
2 & 1
\end{pmatrix}$}
\itemps {Halle una matriz $X$ tal que $X \cdot C=D$, siendo \[ D=\begin{pmatrix}
2 & -2 \\
-2 & 2 \\
4 & 4 
\end{pmatrix}\]}{$D=\begin{pmatrix} -3  & 4 \\ 3 & -4 \\ 2 & 0 \end{pmatrix}$}
\eex
\end{ejer}

\begin{ejer}
Se dice que una matriz cuadrada $A$ es idempotente si se cumple que $A^2=A$
\bex
\itemps {Si $A$ es una matriz idempotente, calcule razonadamente $A^{2015}$}{$A$}
\itemps {Determine para qué valores de los parámetros $a$ y $B$ la siguiente matriz es idempotente \[ A=\begin{pmatrix}
a & -a & 0 \\
-a & a & 0 \\
0 & 0 & b \end{pmatrix} \]}{•}
\eex
\end{ejer}

\begin{ejer}
Se dice que una matriz cuadrada $A$ es involutiva si cumple $A^2=I$, donde $I$ denota la matriz identidad.
\bex
\itemps {Justifique razonadamente que toda matriz involutiva es regular (o invertible)}{$A^{-1}\cdot A^2= A^{-1}\cdot I \rightarrow A=A^{-1}$}
\itemps {Determine para qué valores de los parámetros $a$ y $b$ la siguiente matriz es involutiva:
\[ \begin{pmatrix}
a & a & 0 \\
a & -a & 0 \\
0 & 0 & b
\end{pmatrix} \]}{$a=\pm \sqrt{\dfrac{1}{2}} \qquad b=\pm 1 $}
\eex
\end{ejer}
\begin{ejer}
Dada la matriz $A=\begin{pmatrix}
x & -2 \\
5 & -x
\end{pmatrix} $, calcular qué valor debe tener $x$ para que la matriz inversa de $A$ coincida con la opuesta de $A$ (esto es, $A^{-1}=-A$).
\begin{solu}
$x=\pm 3 $
\end{solu}
\end{ejer}

\begin{ejer}
Se considera la matriz: $A = \begin{pmatrix}
1 & -2 & 0 \\
-2 & 2 & -1 \\
0 & 1 & 1
\end{pmatrix}$
\begin{enumerate}[label=\alph*)]
\item Razone si la matriz $A$ es simétrica.
\item Calcule $A^{-1}$.
\item Resuelva la ecuación matricial $2 \cdot X \cdot A - A^2 -3 \cdot I = O $.
\end{enumerate}

\begin{solu}
\begin{tasks}(3)[label=\alph*)]
\task No, ya que $A \neq A^t$
\task $A^{-1}=\begin{pmatrix}
-3 & -2 & -2 \\
-2 & -1 & -1 \\
2 6 1 & 2
\end{pmatrix}$
\task $X=\begin{pmatrix}
-1 & -2 & -1 \\
-2 & \frac{1}{2} & -1 \\
1 & 1 & \frac{3}{2}
\end{pmatrix}$
\end{tasks}
\end{solu}
\end{ejer}


\newpage

\soluciones
\solucionesCap{\thechapter}