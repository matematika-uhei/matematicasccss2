%--------------------------------------------------------------------------
% ARCHIVO .TEX DE DISEÑO
% PAQUETES Y ESTILO DEL LIBRO 
%--------------------------------------------------------------------------
% Paquetes 
\usepackage[spanish,es-noshorthands]{babel}
\usepackage[utf8]{inputenc}                      % Entrada de acentos
\usepackage[T1]{fontenc}
\usepackage[autostyle, spanish = mexican]{csquotes}% manejo de comillas: " "
\MakeOuterQuote{"}
%--------------------------------------------------------%
\usepackage{pslatex}                              % Fuentes finas postscript
\usepackage[sc]{mathpazo}                         % Fuentes mathpazo
%--------------------------------------------------------%
\usepackage{helvet}
\linespread{1.05}   
\renewcommand{\baselinestretch}{1} 
\renewcommand{\arraystretch}{1.5}                              % Fuente Palatino necesita espaciado
%\usepackage[full]{textcomp}                        % Caracteres especiales como ' (recto)
\usepackage{xcolor}                                % Color: X11names (en el documentclass)
\usepackage[scaled=0.83]{beramono}
\usepackage{chancery}

\usepackage{sagetex}

% COLORES personales---------------------------------------------------
\definecolor{styrmitcrverdecomentario}{rgb}{0,0.5019,0}
    \definecolor{colortitulo}{RGB}{0,0,122} % 
    \definecolor{colordominante}{RGB}{0,0,122}
    \definecolor{colordominanteF}{RGB}{68,0,170}
    \definecolor{colordominanteD}{RGB}{137,46,55}
    \definecolor{mostaza}{RGB}{231,196,25}
    \definecolor{amarilloM}{RGB}{248,199,90}
    \definecolor{amarilloD}{RGB}{251,237,121}
    \definecolor{azulF}{rgb}{.0,.0,.3}
    \definecolor{grisD}{rgb}{.3,.3,.3}
    \definecolor{grisF}{rgb}{.6,.6,.6}
    \definecolor{grisamarillo}{RGB}{248,248,245} 
    \definecolor{miverde}{RGB}{59,93,43}
    \definecolor{ocre}{RGB}{59,93,43}
    \definecolor{verdep}{RGB}{139,179,106}
    \definecolor{verdencabezado}{RGB}{166,206,58}
    \definecolor{verdeF}{RGB}{101,152,59}
    \colorlet{mygray}{black!20}
  	\newcommand{\nverde}{\color{styrmitcrverdecomentario}}
	\newcommand{\wverde}{\color{styrmitcrverdecomentario}}
	\newcommand{\wVerdeComentario}{\color{styrmitcrverdecomentario}}
	\newcommand{\wcvc}{\color{styrmitcrverdecomentario}}
    \definecolor{azulF}{rgb}{.0,.0,.3}
     \newcommand{\azulf}{\color{azulF}}
    \newcommand{\verde}{\color{miverde}}
    \definecolor{morado}{RGB}{80,12,207} % 
     \newcommand{\mimorado}{\color{morado}}
     \definecolor{moradoB}{RGB}{136,0,170}
     \newcommand{\wmora}[1]{{\color{moradoB} #1}}
% Fin COLORES personales-------------------------------------------------
%\usepackage{psboxit}
\usepackage{pstricks,fontawesome} %fontawesome for eyes
\usepackage{xparse}
\usepackage{tcolorbox} 
\tcbuselibrary{skins,breakable} % Librerías tcolorbox
\usepackage{tikz}% Cajas de Teoremas, ejemplos, etc.
\usetikzlibrary{positioning,shadows,backgrounds,calc}%
\usepackage{tikzpagenodes}
\usepackage{tikz-3dplot}
\usetikzlibrary{matrix}
\usetikzlibrary{calendar,decorations.markings} 
\usetikzlibrary{shapes}
\usepackage{tkz-tab,tkz-euclide,tkz-fct}
\usetkzobj{all}

\usepackage{xargs}                                 % Comandos con opciones
\DeclareGraphicsExtensions{.pdf,.png,.jpg, .eps}
\usepackage{multicol}
% %\usepackage{epstopdf}% Conversión - Miktes 2.9 o inferior, TexLive 2009. o inferior
\usepackage[small,bf]{caption}
\usepackage[breaklinks,colorlinks=true, pdfstartview=FitV, linkcolor=azulF, citecolor=azulF, urlcolor=azulF]{hyperref}
\usepackage{amsmath,amssymb,amsfonts,latexsym,cancel,stmaryrd,amsthm,extarrows,xfrac}%
\usepackage[ruled,,vlined,lined,linesnumbered,algochapter]{algorithm2e}
\usepackage{framed}
\usepackage{titletoc}
\usepackage{calc}
\usepackage{colortbl} 
\usepackage{tabularx}
\newcounter{lasfilas}
\renewcommand\thelasfilas{\alph{lasfilas}}
\newenvironment{enumfilas}[1]
{\setcounter{lasfilas}{0}
\par\noindent\tabularx{\linewidth}[t]
{*{#1}{>{\stepcounter{lasfilas}\makebox[1.8em][l]{\thelasfilas)\hfill}}X}} %
}{\endtabularx}
\usepackage{fancyvrb}
\usepackage{array}
\usepackage{wasysym}
\usepackage{xtab}
\usepackage{booktabs}
\usepackage[shortlabels]{enumitem}
\usepackage{textgreek}% ver abajo la definición de los contadores griegos
%\usepackage{enumerate}
\SetEnumitemKey{:roman}{label=\textup{(\roman*)}}
\usepackage{esvect}

\renewcommand{\thempfootnote}{\arabic{mpfootnote}} %footnote con n\'umeros

%----------------------------------------------------------------------------------------
% Fuentes
%----------------------------------------------------------------------------------------
% Comandos para fuentes especiales
\newcommandx*{\fnte}[4][1=pag,2=9,3=n]{{\color{azulF}\fontfamily{#1}
\fontsize{#2}{1}\fontshape{#3}\selectfont{#4}}}

\newcommandx*{\fntb}[4][1=pag,2=11,3=n]{{\color{azulF}\fontfamily{#1}\fontsize{#2}{1}\fontseries{b}\fontshape{#3}\selectfont{#4}}}

\newcommandx*{\fntg}[4][1=pag,2=9,3=n]{{\color{grisF}\fontfamily{#1}\fontsize{#2}{1}\fontshape{#3}\selectfont{#4}}}

\newcommand{\fhv}[2]{{\fontfamily{pag}\fontsize{#1}{1}\selectfont{#2}}}

\newcommand{\fhvb}[2]{{\fontfamily{pag}\fontseries{b}\fontsize{#1}{1}\selectfont{#2}}}
% Fin fuentes----------------------------------------------------------

%********************************** DISENO *************************************

%----------------------------------------------------------------------------------------
% Cabeceras
%----------------------------------------------------------------------------------------
%-----------------Cabeceras--------------------------------------------------------

\newcommand{\helv}{\color{azulF}\fontfamily{phv}\fontsize{8}{11}\selectfont}
\usepackage{fancyhdr}
\pagestyle{fancy}
 \renewcommand{\chaptermark}[1]{\markboth{#1}{#1}}
 \fancyhf{} % borra cabecera y pie actuales
 \fancyhead[LE,RO]{\bfseries \helv\thepage} %Left Even page - Right Odd page
 %\fancyhead[R]{\bfseries \helv\thepage}  % Numeración siempre a la derecha
 %\fancyhead[LO]{\helv\rightmark}
 %\fancyhead[RE]{\helv\leftmark}
 \fancyhead[LO,LE]{\helv \rightmark }

 \renewcommand{\headrulewidth}{0.5pt} % Sin raya. Con raya?: cambiar {0} por {0.5pt}
 \renewcommand{\footrulewidth}{0pt}
 \setlength\headheight{14.5pt}
 \fancyheadoffset[R]{0.0cm} %Numeración de página en el borde de la página
  \addtolength{\headheight}{0.5pt} % espacio para la raya
 \fancypagestyle{plain}{%
 \fancyhead{} % elimina cabeceras y raya en páginas "plain"
 \renewcommand{\headrulewidth}{0pt} }
% Fin cabeceras -------------------------------------------------------------------



%--------------------------------------------------------stiloLatex.tex--------------------------------
% Título
%----------------------------------------------------------------------------------------
\newcommand*{\titulo}[4]{\begingroup%
\raggedleft 
\vspace*{\baselineskip} % Espacio en blanco en la parte superior de la página
{#1}\\[0.167\textheight] % Autor
{ #2}\\[\baselineskip] % pre-título
{\textcolor{colortitulo}{#3}}\\ % Título
{#4}\par % Descripción adicional
\bigskip
%1%

\bigskip
\bigskip
\bigskip
\bigskip


\vfill % Espacio en blanco entre el bloque de título y "la editorial"

{\raggedright
\begin{minipage}[c]{0.15\textwidth}
\raisebox{-1.0cm}{\includegraphics[width=4cm]{images/logoa}}
 \end{minipage}
\  \ \hfill\begin{minipage}[t]{0.6\textwidth}
{\color{gray}
 \fhvb{9}{\color{azulF}Departamento de Matemáticas.}
  \fntg[pag][8]{\color{grisF}
  \href{https://ived.eus}{(https://ived.eus).}}}
\end{minipage}          
}%raggedright
\vspace*{3\baselineskip} % Espacio en blanco antes del final de página
\endgroup}
% Fin Titulo--------------------------------------------------------


%----------------------------------------------------------------------------------------
%	MAIN TABLE OF CONTENTS
%----------------------------------------------------------------------------------------

\usepackage{titletoc} % Required for manipulating the table of contents

\contentsmargin{0cm} % Removes the default margin

% Part text styling
\titlecontents{part}[0cm]
{\addvspace{20pt}\centering\large\bfseries}
{}
{}
{}

% Chapter text styling
\titlecontents{chapter}[1.25cm] % Indentation
{\addvspace{12pt}\large\sffamily\bfseries} % Spacing and font options for chapters
{\color{ocre!60}\contentslabel[\Large\thecontentslabel]{1.25cm}\color{ocre}} % Chapter number
{\color{ocre}}  
{\color{ocre!60}\normalsize\;\titlerule*[.5pc]{.}\;\thecontentspage} % Page number

% Section text styling
\titlecontents{section}[1.25cm] % Indentation
{\addvspace{3pt}\sffamily\bfseries} % Spacing and font options for sections
{\contentslabel[\thecontentslabel]{1.25cm}} % Section number
{}
{\hfill\color{azulF}\thecontentspage} % Page number
[]

% Subsection text styling
\titlecontents{subsection}[1.25cm] % Indentation
{\addvspace{1pt}\sffamily\small} % Spacing and font options for subsections
{\contentslabel[\thecontentslabel]{1.25cm}} % Subsection number
{}
{\ \titlerule*[.5pc]{.}\;\thecontentspage} % Page number
[]

% List of figures
\titlecontents{figure}[0em]
{\addvspace{-5pt}\sffamily}
{\thecontentslabel\hspace*{1em}}
{}
{\ \titlerule*[.5pc]{.}\;\thecontentspage}
[]

% List of tables
\titlecontents{table}[0em]
{\addvspace{-5pt}\sffamily}
{\thecontentslabel\hspace*{1em}}
{}
{\ \titlerule*[.5pc]{.}\;\thecontentspage}
[]

%----------------------------------------------------------------------------------------
%	MINI TABLE OF CONTENTS IN PART HEADS
%----------------------------------------------------------------------------------------

% Chapter text styling
\titlecontents{lchapter}[0em] % Indenting
{\addvspace{15pt}\large\sffamily\bfseries} % Spacing and font options for chapters
{\color{ocre}\contentslabel[\Large\thecontentslabel]{1.25cm}\color{ocre}} % Chapter number
{}  
{\color{ocre}\normalsize\sffamily\bfseries\;\titlerule*[.5pc]{.}\;\thecontentspage} % Page number

% Section text styling
\titlecontents{lsection}[0em] % Indenting
{\sffamily\small} % Spacing and font options for sections
{\contentslabel[\thecontentslabel]{1.25cm}} % Section number
{}
{}

% Subsection text styling
\titlecontents{lsubsection}[.5em] % Indentation
{\normalfont\footnotesize\sffamily} % Font settings
{}
{}
{}

%----------------------------------------------------------------------------------------
% CAPITULO Estilo simple
%----------------------------------------------------------------------------------------
\usepackage[explicit]{titlesec}                                                 
    \usepackage{titletoc}
\usepackage{epigraph}
\usepackage{xpatch}


\newlength\ChapWd
\settowidth\ChapWd{\huge\chaptertitlename}

\definecolor{myblue}{RGB}{0,0,122}

\titleformat{\chapter}[display]
  {\normalfont\filleft\sffamily}
  {\tikz[remember picture,overlay]
    {
    \node[fill=miverde,font=\fontsize{60}{72}\selectfont\color{white},anchor=north east,minimum size=\ChapWd] 
      at ([xshift=-2.5cm,yshift=-50pt]current page.north east) 
      (numb) {\thechapter};
    \node[rotate=90,anchor=south,inner sep=0pt,font=\huge] at (numb.west) {\chaptertitlename};
    }
  }{0pt}{\fontsize{33}{40}\selectfont\color{miverde}#1}[]
\titlespacing*{\chapter}
  {0pt}{0pt}{10pt}


\makeatletter
\xpatchcmd{\ttl@printlist}{\endgroup}{{\noindent\color{myblue}\rule{\textwidth}{1.5pt}}\vskip30pt\endgroup}{}{}
\makeatother

\newcommand\DoPToC{%
\startcontents[chapters]
\printcontents[chapters]{}{1}{\noindent{\color{miverde}\rule{\textwidth}{1.5pt}}\par\medskip}%
}

\setlength\epigraphrule{0pt}
\renewcommand\textflush{flushright}
\renewcommand\epigraphsize{\normalsize\itshape}


%-------------------CONTENIDO -----------------------------------------------------
%\usepackage{kpfonts}
%\usepackage{titletoc}
\contentsmargin{0cm}
\titlecontents{chapter}[0pc]
{\addvspace{30pt}%
\begin{tikzpicture}[remember picture, overlay]%
\draw[fill=white,draw=white] (-4,-.1) rectangle (0.0,0.5);% rectángulo antes del "capítulo tal"
\pgftext[left,x=-1.5 cm,y=0.2cm]{\color{verdeF}\Huge\sc\bfseries \ \thecontentslabel};%
\end{tikzpicture}\color{verdeF}\large\sc\bfseries}%
{}
{}
{\;\titlerule\;\large\sc\bfseries Página \thecontentspage
\begin{tikzpicture}[remember picture, overlay]
\draw[fill=white,draw=verdeF] (2pt,0) rectangle (6,0.1pt);
\end{tikzpicture}}%
\titlecontents{section}[2.4pc]
{\addvspace{1pt}}
{\contentslabel[\color{azulF}\thecontentslabel]{2.4pc}}
{}
{\hfill\small \color{azulF}\thecontentspage}
[]
\titlecontents{subsection}[4.8pc]
{\addvspace{1pt}\small}
{\contentslabel[\color{azulF}\thecontentslabel]{2.4pc}}
{}
{\hfill\small \color{azulF}\thecontentspage}
[]

\makeatletter
\renewcommand{\tableofcontents}{%
%Título: Indice General
\chapter*{\contentsname}%
\@starttoc{toc}}
\makeatother

% Fin Contenido -------------------------------------------------------------------





%----------------------------------------------------------------------------------------
%	Numeración de las secciones -- en el margen
%----------------------------------------------------------------------------------------
%----------------------------------------------------------------------------------------
%	Numeración de las secciones -- en el margen
%----------------------------------------------------------------------------------------
%--Subsecciones-subsubsecciones-paragraph...apéndice------------------------------------
%  TEORIA
% \@startsection {NAME}{LEVEL}{INDENT}{BEFORESKIP}{AFTERSKIP}{STYLE} 
%            optional * [ALTHEADING]{HEADING}
%    Generic command to start a section.  
%    NAME       : e.g., 'subsection'
%    LEVEL      : a number, denoting depth of section -- e.g., chapter=1,
%                 section = 2, etc.  A section number will be printed if
%                 and only if LEVEL < or = the value of the secnumdepth
%                 counter.
%    INDENT     : Indentation of heading from left margin
%    BEFORESKIP : Absolute value = skip to leave above the heading.  
%                 If negative, then paragraph indent of text following 
%                 heading is suppressed.
%    AFTERSKIP  : if positive, then skip to leave below heading,
%                       else - skip to leave to right of run-in heading.
%    STYLE      : commands to set style
%  If '*' missing, then increments the counter.  If it is present, then
%  there should be no [ALTHEADING] argument.  A sectioning command
%  is normally defined to \@startsection + its first six arguments.
%
%\RequirePackage{appendix}

\makeatletter
\renewcommand{\@seccntformat}[1]{\llap{\textcolor{miverde}{\csname the#1\endcsname}\hspace{1em}}}                    
\renewcommand{\section}{\@startsection{section}
{1}{0.9cm}%0.9 corre la numeración hacia adentro
{-4ex \@plus -1ex \@minus -.4ex}
{1ex \@plus.2ex }
{\color{miverde}\normalfont\Large\sffamily\bfseries}}
\renewcommand{\subsection}{\@startsection {subsection}{2}{1.1cm}
{-3ex \@plus -0.1ex \@minus -.4ex}
{0.5ex \@plus.2ex }
{\color{miverde}\normalfont\large\sffamily\bfseries}}
\renewcommand{\subsubsection}{\@startsection {subsubsection}{3}{\z@}
{-2ex \@plus -0.1ex \@minus -.2ex}
{0.2ex \@plus.2ex }
{\color{miverde}\normalfont\small\sffamily\bfseries}}                        
\renewcommand\paragraph{\@startsection{paragraph}{4}{\z@}
{-2ex \@plus-.2ex \@minus .2ex}
{0.1ex}
{\normalfont\small\sffamily\bfseries}}
\makeatother
% Fin numeración secciones



%---------------------------------------------------------------------------------
%  Entornos:  Ejemplo, teorema, proposición, lema, lista de ejercicios, 
%             caja interludio, caja simple  
%---------------------------------------------------------------------------------

%  Cajas con el paquete  tcbcolor
%  CONTADORES: ejemplo, definicion, lema, teorema, corolario, proposicion,ejercicio 
\newcounter{tcbteo}[chapter]
\renewcommand{\thetcbteo}{\thechapter.\arabic{tcbteo}}

\newcounter{tcbdefi}[chapter]
\renewcommand{\thetcbdefi}{\thechapter.\arabic{tcbdefi}}

\newcounter{tcblema}[chapter]
\renewcommand{\thetcblema}{\thechapter.\arabic{tcblema}}

\newcounter{tcbcoro}[chapter]
\renewcommand{\thetcbcoro}{\thechapter.\arabic{tcbcoro}}

%  \newcounter{tcbListaEjercicios}[chapter]
%  \renewcommand{\thetcbListaEjercicios}{\thechapter.\arabic{tcbListaEjercicios}}

\newcounter{tcbpropo}[chapter]
\renewcommand{\thetcbpropo}{\thechapter.\arabic{tcbpropo}}

\newcounter{tcbvoca}[chapter]
\renewcommand{\thetcbvoca}{\thechapter.\arabic{tcbvoca}}

\newlength{\examlen}
\tikzset{
    wnodeTeorema/.style={%
         rectangle,  top color=gray!5, bottom color=gray!5,
         inner sep=1mm,anchor=west,font=\small\bf\sffamily},
   wnodeminimo/.style={%
         rectangle,  top color=white, bottom color=white,
         text=azulF,inner sep=1mm,anchor=west,font=\small\bf\sffamily}      
}



%\begin{teorema}  o \begin{teorema}[de tal] o \begin{teorema}[][ref]
% Teorema -----------------------------------------------------
\newtcolorbox{wwteorema}[3][]{%
arc=0mm,breakable,enhanced,colback=gray!5,boxrule=0pt,top=7mm,drop fuzzy shadow,
fontupper=\normalsize,step and label={tcbteo}{#3},
overlay unbroken = {\draw[color=colordominanteF,line width=0.2pt] (frame.north west)--([xshift=0pt]frame.north east);
%Caja de Título: teo --
\node[wnodeTeorema](tituloteo) at ([xshift=0pt, yshift=-4mm]frame.north west)
{\textbf{\color{colordominanteF} Teorema \thetcbteo \;#2}};
%Borde superior --
\draw[colordominanteF,line width=2.5cm] ([xshift=1.25cm, yshift=0cm]frame.north west)-- +(\examlen,3pt);
},%
overlay first = {\draw[color=colordominanteF,line width=0.2pt] (frame.north west)--([xshift=0pt]frame.north east);
%Caja de Título: teo --
\node[wnodeTeorema](tituloteo) at ([xshift=0pt, yshift=-4mm]frame.north west)
{\textbf{\color{colordominanteF} Teorema \thetcbteo \;#2}};
%Borde superior --
\draw[colordominanteF,line width=2.5cm] ([xshift=1.25cm, yshift=0cm]frame.north west)-- +(\examlen,3pt);
},%
% Mantener borde en cambio de página 
overlay last = {\draw[color=colordominanteF,line width=0.2pt] (frame.south west)--([xshift=0pt]frame.south east);
                } 
#1}
%-
\NewDocumentEnvironment{teorema}{O{} O{} O{}}{\smallskip\begin{wwteorema}{#1}{#2}%
 #3}{\end{wwteorema}\smallskip }
% TEOREMA---------------------------------------------------------



%\begin{proposicion}  o \begin{proposicion}[de tal] o \begin{proposicion}[][ref]
% Proposición-----------------------------------------------------
\newtcolorbox{wwpropo}[3][]{%
arc=0mm,breakable,enhanced,colback=gray!5,boxrule=0pt,top=7mm,drop fuzzy shadow,
fontupper=\normalsize,step and label={tcbpropo}{#3},
overlay first ={\draw[color=colordominante,line width=0.2pt] (frame.north west)--([xshift=0pt]frame.north east);
%Caja de Título: propo --
\node[wnodeTeorema](tituloteo) at ([xshift=0pt, yshift=-4mm]frame.north west)
{\textbf{\color{colordominante} Proposición \thetcbpropo\;#2}};
%Borde superior --
\draw[colordominanteF,line width=2.5cm] ([xshift=1.25cm, yshift=0cm]frame.north west)-- +(\examlen,3pt);
}, %
overlay first ={\draw[color=colordominante,line width=0.2pt] (frame.north west)--([xshift=0pt]frame.north east);
%Caja de Título: propo --
\node[wnodeTeorema](tituloteo) at ([xshift=0pt, yshift=-4mm]frame.north west)
{\textbf{\color{colordominante} Proposición \thetcbpropo\;#2}};
%Borde superior --
\draw[colordominanteF,line width=2.5cm] ([xshift=1.25cm, yshift=0cm]frame.north west)-- +(\examlen,3pt);
}, %
% Mantener borde en cambio de página 
overlay last = {\draw[color=colordominanteF,line width=0.2pt] (frame.south west)--([xshift=0pt]frame.south east);
                } 
#1}
%-
\NewDocumentEnvironment{proposicion}{O{} O{} O{}}{\smallskip\begin{wwpropo}{#1}{#2}%
 #3}{\end{wwpropo}\smallskip }
% ---------------------------------------------------------



% LEMA -----------------------------------------------------------
 \newtcolorbox{wwlema}[3][]{%
arc=0mm,breakable,enhanced,colback=gray!5,boxrule=0pt,drop fuzzy shadow,
top=1mm, left=3pt,
step and label={tcblema}{#3},
fontupper={\small\bf\sffamily {\color{azulF}Lema \thetcblema \;#2}}~\normalfont, %"Lema..."+texto del cuerpo
overlay unbroken  = {%barra vertical
\draw[color=gray,line width=3pt] ([xshift=2pt] frame.north west)--([xshift=2pt] frame.south west);              
         },%
overlay first  = {%barra vertical
\draw[color=gray,line width=3pt] ([xshift=2pt] frame.north west)--([xshift=2pt] frame.south west);              
         },%
% Mantener borde en cambio de página     
overlay last ={\draw[color=gray,line width=3pt] ([xshift=2pt] frame.north west)--([xshift=2pt] frame.south west);   },    
#1}
%-
\NewDocumentEnvironment{lema}{O{} O{} O{}}{\smallskip\begin{wwlema}{#1}{#2}%
#3}{\end{wwlema}\smallskip }
%LEMA--------------------------------------------------------------


% % Corolario -------------------------------------------------------
\newtcolorbox{wwcoro}[2][]{%
arc=0mm,breakable,enhanced,colback=gray!5,boxrule=0pt,drop fuzzy shadow,
top=1mm,left=3pt,
fontupper={\small\bf\sffamily {\color{azulF}Corolario \thetcbcoro}\;}~\normalfont, 
step and label={tcbcoro}{#2},
overlay unbroken  = {%barra vertical
\draw[color=gray,line width=3pt] ([xshift=2pt] frame.north west)--([xshift=2pt] frame.south west);                
        },%
overlay first  = {%barra vertical
\draw[color=gray,line width=3pt] ([xshift=2pt] frame.north west)--([xshift=2pt] frame.south west);                
        },%
% Mantener borde en cambio de página     
overlay last ={\draw[color=gray,line width=3pt] 
                     ([xshift=2pt] frame.north west)--([xshift=2pt] frame.south west);
              }     
#1}
%-
\NewDocumentEnvironment{corolario}{O{} O{}}{\smallskip\begin{wwcoro}{#1}%
}{\end{wwcoro}\smallskip }
% Corolario------------------------------------------------------


% % Definición---------------------------------------------------
\newtcolorbox{wwdefinicion}[3][]{%
arc=0mm,breakable,enhanced,colback=gray!5,boxrule=0pt,drop fuzzy shadow,
top=6mm,fontupper=\normalsize,step and label={tcbdefi}{#3},
overlay unbroken  = {
%barra vertical
\draw[color=miverde,line width=3pt] ([xshift=2pt] frame.north west)--([xshift=2pt] frame.south west);        
%Caja de Título: defi --
\node[wnodeTeorema](titulodefi) at ([xshift=4.5pt, yshift=-3mm]frame.north west)
{\textbf{\fhv{12}{#2}}};
                }, %overlay
overlay first  = {
%barra vertical
\draw[color=miverde,line width=3pt] ([xshift=2pt] frame.north west)--([xshift=2pt] frame.south west);        
%Caja de Título: defi --
\node[wnodeTeorema](titulodefi) at ([xshift=4.5pt, yshift=-3mm]frame.north west)
{\textbf{\fhv{12}{#2}}};
                }, %overlay
% Mantener borde en cambio de página
overlay last    = {%barra vertical
\draw[color=miverde,line width=3pt] ([xshift=2pt] frame.north west)--([xshift=2pt] frame.south west);}
#1}
%-
\NewDocumentEnvironment{definicion}{O{} O{} O{}}{\smallskip\begin{wwdefinicion}{#1}{#2}%
 #3}{\end{wwdefinicion}\smallskip }
% %DEFINICION---------------------------------------------------------


% Caja para Ejemplo --------------------------------------------------
\newcounter{tcbejem}[chapter]
\renewcommand{\thetcbejem}{\thechapter.\arabic{tcbejem}}
\colorlet{colorfondoejemplo}{gray!5}
\definecolor{colorejemplo}{rgb}{.0,.0,.3}
% Ejemplo
\newtcolorbox{wwejemplo}[1]{%
skin=bicolor,breakable, colframe=miverde, colbacktitle=miverde ,fonttitle=\bfseries\color{white},colback=gray!20 ,colbacklower=white,fontupper=\sffamily , step=tcbejem,  title=\Large{Ejemplo \quad  \thetcbejem} \quad #1}
%-
\NewDocumentEnvironment{ejemplo}{O{} }{\smallskip\begin{wwejemplo}{#1}}{\end{wwejemplo}\smallskip }

%Caja para ejemplo a dos columnas
\newtcolorbox{wwejemplocolumna}[1]{%
skin=bicolor,sidebyside, breakable, colframe=miverde, colbacktitle=miverde ,fonttitle=\bfseries\color{white},colback=gray!20 ,colbacklower=white,fontupper=\sffamily , step=tcbejem,  title=\Large{Ejemplo \quad  \thetcbejem} \quad #1}
%-
\NewDocumentEnvironment{ejemplocolumna}{O{} }{\smallskip\begin{wwejemplocolumna}{#1}}{\end{wwejemplocolumna}\smallskip }


%PROBLEMA RESUELTO-----------------------------------------------------------------
% Caja para Ejemplo --------------------------------------------------
\newcounter{tcbresuelto}[chapter]
\renewcommand{\thetcbresuelto}{\thechapter.\arabic{tcbresuelto}}
\colorlet{colorfondoejemplo}{gray!5}
\definecolor{colorejemplo}{rgb}{.0,.0,.3}
% Ejemplo
\newtcolorbox{wwresuelto}{%
arc=0mm, breakable,enhanced, colback=white, colframe=verdep ,fonttitle=\bfseries\color{black},step=tcbresuelto,  title=Ejercicio resuelto \thetcbresuelto }
%-
\NewDocumentEnvironment{resuelto}{O{} }{\smallskip\begin{wwresuelto}{#1}}{\end{wwresuelto}\smallskip }

%%Entorno indice de contenido en capitulo%%
\newtcolorbox{wwindice}{%
 breakable,enhanced, colback=white, colframe=colordominanteD ,fonttitle=\bfseries\color{white}, title=Contenido }
%-
\NewDocumentEnvironment{indice}{O{} }{\smallskip\begin{wwindice}{#1}}{\end{wwindice}\smallskip }
%%Fin entorno indice%%%

% CAJA (interludio, comentario...)---------------------------------------
\definecolor{colrnodocaja}{RGB}{44,91,144}
\definecolor{colrfondocaja}{RGB}{241,241,227}

\newtcolorbox{wwcaja}[2][]{%
arc=0mm,breakable,drop fuzzy shadow,
enhanced,colback=gray!4,
boxrule=0pt,
top=3mm, %Separación vertical - inicia texto
enlarge top by=\baselineskip/2+1mm,
enlarge top at break by=0mm,pad at break=2mm,
fontupper=\normalsize,
%step and label={tcbca}{#3},
%Borde
overlay unbroken={\draw[color=gray!2,line width=0.2pt] (frame.north west)
  --([xshift=0pt]frame.north east)
  --([xshift=0pt]frame.south east)
  --([xshift=0pt]frame.south west)--(frame.north west);
% Caja de Título CAJA
\node[ rectangle, %minimum width=0cm, minimum height=0.0cm,
         top color=gray!30, bottom color=gray!30,
         inner sep=0.5mm,anchor=west,font=\normalsize]at ([xshift=-0.5pt,  yshift=2.30mm]frame.north west){\fhvb{10}{ \bf #2}};
         },
%Borde
overlay first={\draw[color=gray!2,line width=0.2pt] (frame.north west)
  --([xshift=0pt]frame.north east)
  --([xshift=0pt]frame.south east)
  --([xshift=0pt]frame.south west)--(frame.north west);
% Caja de Título CAJA
\node[ rectangle, %minimum width=0cm, minimum height=0.0cm,
         top color=grisamarillo, bottom color=grisamarillo,
         inner sep=0.5mm,anchor=west,font=\normalsize]at ([xshift=-0.5pt,  yshift=2.30mm]frame.north west){\fhvb{10}{ \bf #2}};
         },
%Borde cambio de página
overlay last={\draw[color=gray!2,line width=0.2pt] (frame.north west)
  --([xshift=0pt]frame.north east)
  --([xshift=0pt]frame.south east)
  --([xshift=0pt]frame.south west)--(frame.north west);}
#1}
%-
\NewDocumentEnvironment{caja}{O{} O{}}{\smallskip\begin{wwcaja}{#1}%
 #2}{\end{wwcaja}\smallskip }
% CAJA de comentario


%CAJA simple---------------------------------------------------------------------
\newtcolorbox{wwbox}[1][]{%
arc=0mm,breakable,drop fuzzy shadow,
enhanced,colback=grisamarillo,
boxrule=0pt,
top=3mm, %Separación vertical - inicia texto
enlarge top by=\baselineskip/2+1mm,
enlarge top at break by=0mm,pad at break=2mm,
fontupper=\normalsize,
%step and label={tcbca}{#3},
%Borde
overlay unbroken={\draw[color=grisamarillo,line width=0.5pt] (frame.north west)
  --([xshift=0pt]frame.north east)
  --([xshift=0pt]frame.south east)
  --([xshift=0pt]frame.south west)--(frame.north west);
        },
%Borde
overlay first={\draw[color=agrisamarillo,line width=0.5pt] (frame.north west)
  --([xshift=0pt]frame.north east)
  --([xshift=0pt]frame.south east)
  --([xshift=0pt]frame.south west)--(frame.north west);
        },
%Borde cambio de página
overlay last={\draw[color=grisamarillo,line width=0.5pt] (frame.north west)
  --([xshift=0pt]frame.north east)
  --([xshift=0pt]frame.south east)
  --([xshift=0pt]frame.south west)--(frame.north west);}
#1}

 \newenvironment{scaja}[1][]{\bigskip\begin{wwbox}%
 #1}{\end{wwbox}}	
% Fin CAJA simple

%CAJA vocabulario-------------------------------------------------------
\newtcolorbox{vocabox}[3][]{%
arc=0mm,breakable,enhanced,colback=white,boxrule=0pt,
top=1mm, left=3pt,
step and label={tcbvoca}{#3},
fontupper={\small\bf\sffamily {\color{azulF}Vocabulario \thetcbvoca \;#2}}~\normalfont, %"Vocabulario..."+texto del cuerpo
overlay unbroken  = {%barra vertical
\draw[color=white,line width=3pt] ([xshift=2pt] frame.north west)--([xshift=2pt] frame.south west);                 
         },%
overlay first  = {%barra vertical
\draw[color=white,line width=3pt] ([xshift=2pt] frame.north west)--([xshift=2pt] frame.south west);                 
         },%
% Mantener borde en cambio de página     
overlay last ={\draw[color=white,line width=3pt] ([xshift=2pt] frame.north west)--([xshift=2pt] frame.south west);   },    
#1}
%-
\NewDocumentEnvironment{vocabulario}{O{} O{} O{}}{\smallskip\begin{vocabox}{#1}{#2}%
#3}{\end{vocabox}\smallskip }	
% Fin vocabulario

%CAJA nota-------------------------------------------------------
\newtcolorbox{notabox}[1][]{%
arc=0mm,breakable,
enhanced,colback=white,
boxrule=0pt,
top=3mm, %Separación vertical - inicia texto
left=25pt,
enlarge top by=\baselineskip/2+1mm,
enlarge top at break by=0mm,pad at break=2mm,
fontupper={\begin{tikzpicture}[overlay]
\node[draw=colordominanteF,line width=1pt,circle,fill=white,font=\sffamily\bfseries,inner sep=2pt,outer sep=0pt] at (-15pt,3pt){\textcolor{colordominanteF}{N}};\end{tikzpicture}}~\normalfont,  %"NOTA..."+texto del cuerpo
%Borde y círculo
overlay first={
\draw[color=white,line width=0.5pt] (frame.north west)
  --([xshift=0pt]frame.north east)
  --([xshift=0pt]frame.south east)
  --([xshift=0pt]frame.south west)--(frame.north west);
        },
%Borde y círculo
overlay first={
\draw[color=white,line width=0.5pt] (frame.north west)
  --([xshift=0pt]frame.north east)
  --([xshift=0pt]frame.south east)
  --([xshift=0pt]frame.south west)--(frame.north west);
        },
%Borde cambio de página
overlay last={\draw[color=white,line width=0.5pt] (frame.north west)
  --([xshift=0pt]frame.north east)
  --([xshift=0pt]frame.south east)
  --([xshift=0pt]frame.south west)--(frame.north west);}
#1}
%-
 \newenvironment{nota}[1][]{\bigskip\begin{notabox}%
 #1}{\end{notabox}}	
% Fin nota




%--------------------------------------------------------------------------------
% LISTAS DE EJERCICIOS
%--------------------------------------------------------------------------------
% \newtheoremstyle{exampstyle}
%   {\topsep} % Space above
%   {\topsep} % Space below
%   {} % Body font
%   {} % Indent amount
%   {\bfseries} % Theorem head font
%   {.} % Punctuation after theorem head
%   {.5em} % Space after theorem head
%   {} % Theorem head spec (can be left empty, meaning `normal')

 \setenumerate[1]{label={\arabic*)}} %poner 1)
 
 % Ligas entres pregunta y respuetsa
 
%%\newtheorem{<name>}{<heading>}[<counter>]
%%%will create an environment <name> for a theorem-like structure; the counter for this structure will be subordinated to <counter>. On the other &&&hand, using
%%\newtheorem{<name>}[<counter>]{<heading>}
%%%will create an environment <name> for a theorem-like structure; the counter 
%%%for this structure will share the previously defined <counter> counter.
 
\usepackage{answers}
\newcommand{\imgRespuesta}[1]{%
 \begin{tikzpicture}[baseline={(char.base)}]
   \node (char) {#1};
   \node[draw,line width=0.8mm,circle,color=amarilloM, minimum size=16pt,inner sep=0pt,overlay] (char.center){};
 \end{tikzpicture}
 }
\newcommand{\devolverse}{\raisebox{-10pt}{\includegraphics[scale=0.4]{images/goback.pdf}}}
\newtheoremstyle{numcolor}{}{}{\large}{}{\color{verdeF}\normalfont\large\sffamily\bfseries}{\;\;}{ }{}
\theoremstyle{numcolor}

%\newtheorem{exer}{...}[section]
%%\newtheorem{<name>}{<heading>}[<counter>]
\newtheorem{exer}{% Liga ejercicio a solución y viceversa
%%-----------marca>1.1.1 >---------------------envía>{1.1.1}
  \hspace{-30pt}\hyperlink{exer:\theexer}{\imgRespuesta{{\color{amarilloM}S}}\;\;}\hypertarget{solu:\theexer}{}%
 }[chapter]

\newenvironment{ejer}{\begin{exer}\phantomsection\label{\theexer} \normalfont}{\end{exer}}
%\Newassociation{xxx}{yyy}{zzz}
% where xxx is an environment in the document, 
% and yyy is an environment which will surround the contents of xxx
% when it is written to symbolic file handle zzz
\Newassociation{solu}{Soln}{ans}
% Lo que se quiere -Lee>{1.1.1} --envía>{1.1.1}, #1=1.1.1
% Problema: Lee en subexer k >{{{k)}}} y envía>{{{1)}}} = pág 1
% Corrección: Leer long string {{{k)}}} es 1  pero long strig {1.1.1} es >1
\usepackage{xstring}  
    \newcommand{\LeeNumEjercicio}[1]{
       \StrLen{1}[\sOne] %\StrLen no es número, es String. arg [\sOne] se usa para almacenar como número
       \StrLen{#1}[\sTwo]
       \ifthenelse{\sOne<\sTwo}{
        \par\bigskip\noindent{\hspace{-30pt}\bfseries\hyperlink{solu:#1}{ \imgRespuesta{{\color{amarilloM}E}}\hspace{12pt} #1} \hypertarget{exer:#1}{}}\quad
       }{ 
         #1
       }
    }
\renewenvironment{Soln}[1]{\LeeNumEjercicio{#1} \;\;}


%  Agregar nombre de sección o subseccion de ejercicios en respuestas:
%  Uso: 
%% \section{Limites}
%% \seccionDeEjercicios{Limites}
%% \subsection*{Lim trig}
%% \subseccionDeEjercicios{Lim trig}
\newcommand\mychaptername{}
% to store the section name
\newcommand\mysectionname{}
\newcommand\mysubsectionname{}
\newcommand{\seccionDeEjercicios}[1]{%
  \Writetofile{ans}{\protect\paragraph*{%
  {\large\azulf\hspace{-1mm}\thesection.\mysectionname\space  #1}
                                       }   }}
\newcommand{\subseccionDeEjercicios}[1]{%
  \Writetofile{ans}{\protect\paragraph*{%
  {\azulf\mysubsectionname\space  #1}
                                       }   }}  

% USO del entorno personalizado---------------------------------------------------
%\begin{ejercicios} --- \end{ejercicios} para listas simples
%\begin{cejercicios} --- \end{cejercicios} para listas en cajas

\NewDocumentEnvironment{ejercicios}{O{}}{%
\bigskip\begin{minipage}{\textwidth}{\bf\verde \fhvb{12}{Ejercicios}}
                                     #1}{\end{minipage}\bigskip}



\colorlet{color1}{gray!5!white}       % color fondo
\definecolor{color2}{RGB}{117,184,68} % color del nodo
% Caja --
\newtcolorbox{wwlistaejercicios}[1][]{%
  arc=0mm,breakable,enhanced,colback=white,boxrule=0pt,top=8mm, 
  enlarge top by=\baselineskip/2+1mm, enlarge top at break by=0mm,
  pad at break=2mm,fontupper=\normalsize,
  overlay unbroken ={ % nodo rectangular  para "Ejercicios"
  \node[rectangle, minimum width=4cm, 
      top color=color2, bottom color=color2, 
      inner sep=1mm,anchor=west,font=\normalsize] 
   at ([xshift=0pt,yshift=-3mm]frame.north west)%
{\textbf{Ejercicios}};},
  overlay first ={ % nodo rectangular  para "Ejercicios"
  \node[rectangle, minimum width=4cm, 
      top color=color2, bottom color=color2, 
      inner sep=1mm,anchor=west,font=\normalsize] 
   at ([xshift=0pt,yshift=-3mm]frame.north west)%
{\textbf{Ejercicios}};},
overlay last ={ } % cambio de página, solo caja gris
#1}

% Nuevo entorno personalizado----------------------------------------
\NewDocumentEnvironment{cejercicios}{O{}}{%
\bigskip\begin{wwlistaejercicios}%
 #1}{\end{wwlistaejercicios}\bigskip } % 
% -------------------------------------------------------------------




% Entorno con Caja para ejercicios-----------------------------------------
%\begin{cajaejercicios}  o \begin{cajaejercicios}[de tal] 
%                        o \begin{cajaejercicios}[][ref]
% Entorno personalizado---------------------------------------------------
\definecolor{colorejercicios}{RGB}{99,42,134}
\newcounter{tcbejer}[chapter]
\renewcommand{\thetcbejer}{\thechapter.\arabic{tcbejer}}

\newtcolorbox{wwejercicios}[3][]{%
arc=0mm,breakable,%drop fuzzy shadow,
enhanced,colback=white,boxrule=0pt,top=7mm,
fontupper=\normalsize,step and label={tcbejer}{#3},
overlay unbroken = {
%Borde grueso superior
\draw[line width=0.2pt] (frame.north west)--([xshift=0pt]frame.north east);
%Caja de Título: Ejer --
\node[ draw=white, top color=white, bottom color=white,inner sep=1mm,anchor=west, font=\small\bf\sffamily](tituloejer) at ([xshift=5mm, yshift=0mm]frame.north west)
{\textbf{\color{miverde}  Ejercicios \thetcbejer \;#2}};
%borde línea inferior
 \draw[color=white,line width=0.2pt] (frame.south west)--([xshift=0pt]frame.south east); 
},%overlay
overlay first = {
%Borde grueso superior
\draw[line width=0.2pt] (frame.north west)--([xshift=0pt]frame.north east);
%Caja de Título: Ejer --
\node[ draw=white, top color=white, bottom color=white, inner sep=1mm,anchor=west, font=\small\bf\sffamily](tituloejer) at ([xshift=5mm, yshift=0mm]frame.north west)
{\textbf{\color{miverde}  Ejercicios \thetcbejer \;#2}};
%borde línea inferior
 \draw[color=white,line width=0.2pt] (frame.south west)--([xshift=0pt]frame.south east); 
},%overlay
% % Mantener borde en cambio de página 
% overlay middle = {\draw[color=colordominante,line width=0.2pt] (frame.north west)--([xshift=0pt]frame.north east);
%                 } 
overlay middle ={},
overlay last = { %borde línea inferior
 \draw[color=white,line width=0.2pt] (frame.south west)--([xshift=0pt]frame.south east); 
                } 
#1}
%-
\NewDocumentEnvironment{cajaejercicios}{O{} O{} O{}}{\smallskip\begin{wwejercicios}{#1}{#2}%
 #3}{\end{wwejercicios}\smallskip }
% ejercicios---------------------------------------------------------


% Comandos para paquete answers
% pregunta-solución
\newcommand{\exersol}[2]{
\begin{ejer} 
#1\scantokens{\begin{solu}#2\end{solu}}
\end{ejer}}
% listas \item pregunta-solución
\newcommand{\itemps}[2]{\item #1\scantokens{\begin{solu} #2\end{solu}}}
%\newcommand{\bex}{\scantokens{\begin{solu} 
%\end{solu}}\begin{enumerate}[label=\alph*.)]}
\newcommand{\bex}{\scantokens{\begin{solu}  \end{solu}}\begin{enumerate}}
\newcommand{\eex}{\end{enumerate}}
%\begin{sol} \end{sol}



% Ejercicio - Caja para ejercicio solitario

\newcounter{tcbejercicio}[chapter]
\renewcommand{\thetcbejercicio}{\thechapter.\arabic{tcbejercicio}}

\newtcolorbox{wwejercicio}[1][]{%
arc=0mm,breakable,enhanced,colback=gray!5,boxrule=0pt,
top=1mm,left=3pt,
fontupper={\bf\sffamily {\color{miverde} \thetcbejercicio .}\;}~\normalfont, 
overlay unbroken  = {%barra vertical
\draw[color=miverde,line width=3pt] ([xshift=2pt] frame.north west)--([xshift=2pt] frame.south west);                 
        },%
overlay first  = {%barra vertical
\draw[color=gray,line width=3pt] ([xshift=2pt] frame.north west)--([xshift=2pt] frame.south west);                 
        },%
% Mantener borde en cambio de página     
overlay last ={\draw[color=gray,line width=3pt] 
                     ([xshift=2pt] frame.north west)--([xshift=2pt] frame.south west);
              }     
#1}
%-
\NewDocumentEnvironment{ejercicio}{O{} O{}}{\smallskip\begin{wwejercicio}{#1}%
}{\end{wwejercicio}\smallskip }
% Ejercicio------------------------------------------------------



% -- Soluciones al final del documento----------------------------
\def\soluciones{
\expandafter\ifx\csname Closesolutionfile\endcsname \relax\else
\Closesolutionfile{ans}\fi
}
\def\solucionesCap#1{\section*{Soluciones del Cap\'{\i}tulo #1}
%\begin{multicols}{2}
\input{ans#1}
%\end{multicols}
}
% Fin Listas de Ejercicios-------------------------------------------


% Fin mis entornos---------------------------------------------------------------


%---------------------------------------------------------------------------------
%  Código de programas (LaTeX en ese caso)
%---------------------------------------------------------------------------------
% Listings
% Listings
\usepackage{listings}
% Puede usar lstlisting|texto| para código en el texto
\captionsetup[lstlisting]{singlelinecheck=false, margin=0pt, font={sf,sl,footnotesize}}

\usepackage{fancyvrb}


\definecolor{mauve}{rgb}{0.58,0,0.82}
\lstset{ %
  language=R,                     % the language of the code
  basicstyle=\bfseries\ttfamily,
  backgroundcolor=\color{yellow!2!gray!2},
  %numbers=left,                   % where to put the line-numbers
  %numberstyle=\tiny\color{gray},  % the style that is used for the line-numbers
%   stepnumber=1,                   % the step between two line-numbers. If it's 1, each line
%                                   % will be numbered
%   numbersep=5pt,                  % how far the line-numbers are from the code 
  showspaces=false,               % show spaces adding particular underscores
  showstringspaces=false,         % underline spaces within strings
  showtabs=false,                 % show tabs within strings adding particular underscores
  %frame=single,                   % adds a frame around the code
  %rulecolor=\color{black},        % if not set, the frame-color may be changed on line-breaks within not-black text (e.g. commens (green here))
  tabsize=2,                      % sets default tabsize to 2 spaces
  captionpos=t,                   % sets the caption-position to bottom
  breaklines=true,                % sets automatic line breaking
  breakatwhitespace=false,        % sets if automatic breaks should only happen at whitespace
  title=\lstname,                 % show the filename of files included with \lstinputlisting;
  frame=lines,
  rulecolor=\color{gray!60},
  framerule=0.1pt, 
                                  % also try caption instead of title
  keywordstyle=\color{blue},      % keyword style
  commentstyle=\color{brown},   % comment style
  stringstyle=\color{mauve},         % string literal style 
  %escapeinside={\%*}{*)},         % if you want to add a comment within your code
  morekeywords={*,...}            % if you want to add more keywords to the set
}

%--- Nombre corto-------------------------------------------------------
%----
\lstnewenvironment{code}
  {\lstset{ %
  language=R,                     % the language of the code
  basicstyle=\bfseries\ttfamily,
  backgroundcolor=\color{yellow!2!gray!2},
  %numbers=left,                   % where to put the line-numbers
  %numberstyle=\tiny\color{gray},  % the style that is used for the line-numbers
%   stepnumber=1,                   % the step between two line-numbers. If it's 1, each line
%                                   % will be numbered
%   numbersep=5pt,                  % how far the line-numbers are from the code 
  showspaces=false,               % show spaces adding particular underscores
  showstringspaces=false,         % underline spaces within strings
  showtabs=false,                 % show tabs within strings adding particular underscores
  %frame=single,                   % adds a frame around the code
  %rulecolor=\color{black},        % if not set, the frame-color may be changed on line-breaks within not-black text (e.g. commens (green here))
  tabsize=2,                      % sets default tabsize to 2 spaces
  captionpos=t,                   % sets the caption-position to bottom
  breaklines=true,                % sets automatic line breaking
  breakatwhitespace=false,        % sets if automatic breaks should only happen at whitespace
  title=\lstname,                 % show the filename of files included with \lstinputlisting;
  frame=lines,
  rulecolor=\color{gray!60},
  framerule=0.1pt, 
                                  % also try caption instead of title
  keywordstyle=\color{blue},      % keyword style
  commentstyle=\color{brown},   % comment style
  stringstyle=\color{mauve},         % string literal style 
  %escapeinside={\%*}{*)},         % if you want to add a comment within your code
  morekeywords={*,...}            % if you want to add more keywords to the set
}%fin lst
  }{}
%------------------------------------------------------------------------

  
%---Encabezado- de listing
\newcommand{\cajaclr}[1]{\color{azulF}$\blacksquare\;$ \fontsize{12}{10}\ttfamily\fontseries{b}\fontshape{n}\selectfont{#1}}
\renewcommand{\lstlistingname}{\cajaclr{Código}}
\newcommand{\wRR}{\lstinline} 




%---------------------------------------------------------------------------------
%  Listas con Tikz 
%---------------------------------------------------------------------------------
% Puntos Tikz y  Enumerate con  Tikz 

  \newcommand{\tpto}{\tikz \shadedraw [shading=ball] (0,0) circle (.1cm);}
  \newcommand{\ttpto}[1]{\begin{tikzpicture}
  	\node[scale=.5, circle, shade, ball color=blue]  {\color{white}\Large\textbf#1};
  \end{tikzpicture}}
  \newcommand{\ttptov}{\begin{tikzpicture}
  	\node[scale=.5, circle, shade, ball color=green]  {\color{white}\Large\textbf.};
  \end{tikzpicture}}
  \newcommand{\ttptor}{\begin{tikzpicture}
  	\node[scale=.5, circle, shade, ball color=red]  {\color{white}\Large\textbf.};
  \end{tikzpicture}}
  \newcommand{\ptoazul}[1]{\begin{tikzpicture}
  	\node[scale=.5, circle, shade, ball color=blue]  {\color{white}\Large\textbf#1};
  \end{tikzpicture}}
 \newcommand{\ptomb}{\ptoazul$\;$}
 \newcommand{\itempto}{\item[\tpto]}
 \newcommand{\pto}{\tpto}
 \newcommand{\ptom}{\tpto$\;$}
 \newcommand{\ptomv}{\ttptov$\;$}
 \newcommand{\ptomr}{\ttptor$\;$}

 % Listas -- con puntos

 %\usepackage{enumerate}
 \newcommand{\witem}[1]{\item[{\bf #1)}]}
 
 \newcommand*{\itembolasgrises}[1]{%
 \footnotesize\protect\tikz[baseline=-3pt]%
 \protect\node[fill=gray!50,shape=circle,draw,inner sep=1.2pt,line width=0.2mm](n1){#1};}
 \newcommand*{\itembolasazules}[1]{%
 \footnotesize\protect\tikz[baseline=-3pt]%
 \protect\node[scale=.5, circle, shade, ball color=blue]  {\color{white}\Large\bf#1};}
 \newcommand*{\itembolasverdes}[1]{%
 \footnotesize\protect\tikz[baseline=-3pt]%
 \protect\node[scale=.5, circle, shade, ball color=green]  {\color{white}\Large\bf#1};}
 \newcommand*{\itembolasrojas}[1]{%
  \footnotesize\protect\tikz[baseline=-3pt]%
  \protect\node[scale=.5, circle, shade, ball color=red]  {\color{white}\Large\bf#1};}
  
%Comando para listas con bolas	
\newcommand{\beaz}{\begin{enumerate}[label=\itembolasazules{\arabic*}]}
\newcommand{\eeaz}{\end{enumerate}}
\newcommand{\bea}{\begin{enumerate}[label=\alph*.)]}
\newcommand{\eea}{\end{enumerate}}
\newcommand{\cgr}[3]{#1 \equiv #2\;(\mbox{\rm mod}\;#3)}
% ----------------------------------
% Fin de cosas adicionales -


%-----------------------------------------------------------------------------
% TABLAS CON Tikz
%----------------------------------------------------------------------------- 
\usepackage{array}
\usetikzlibrary{calc,fit,shadows,arrows,positioning}
\pgfdeclarelayer{background}
\pgfdeclarelayer{foreground}
\pgfsetlayers{background,main,foreground}
%--
%------------------------------------------------------------------------------
% Data Table
%------------------------------------------------------------------------------
\newsavebox{\dataTableContent} % Box
\newenvironment{dataTable}[1] % \new environment
{%
\begin{lrbox}{\dataTableContent}%
\begin{tabular}{#1}}%
%
{%
\end{tabular}
\end{lrbox}
\begin{tikzpicture}
\node [inner xsep=0pt] (tbl){\usebox{\dataTableContent}};
\begin{pgfonlayer}{background}
% table
\draw[rounded corners=1pt,top color=miverde,bottom color=miverde,draw=black]
(tbl.north east) rectangle (tbl.south west);
% top line
\draw[rounded corners=1pt,top color=gray!10!black,bottom color=gray!50!black,draw=black]%
($(tbl.north west)$) rectangle ($(tbl.north east)-(0,1.5\baselineskip)$);
% bottom rule
\draw[rounded corners=0.25pt,fill=gray,draw=black]%
(tbl.south west) rectangle ($(tbl.south east)+(0,0.05)$);
\end{pgfonlayer}
\end{tikzpicture}}
% --
 
 
 
%----------------------------------------------------------------------------- 
%- Modoficación de chapter para que abra y cierre archivos ans----------------
%-----------------------------------------------------------------------------
% Para escribir en answer sol, begindocument
%-----------------------------------------------------
\newcount\ansj % contador de listas ans i
\ansj=\thechapter
%           % -- Infiltrar \chapter --------------------------------------------
\makeatletter
\let\stdchapter\chapter % \stdchapter guarda la def original de \chapter
\renewcommand*\chapter{ % -- Infiltrar: abrir y cerrar archivos ans j---
\expandafter\ifx\csname Closesolutionfile\endcsname \relax\else
\Closesolutionfile{ans}\fi
\expandafter\ifx\csname Opensolutionfile\endcsname \relax\else
\Opensolutionfile{ans}[ans\number\ansj]\advance\ansj by 1
\fi
                        %------------------------------------------------------
                        % Volver a la def original de \chapter
\@ifstar{\starchapter}{\@dblarg\nostarchapter}}
\newcommand*\starchapter[1]{\stdchapter*{#1}}
\def\nostarchapter[#1]#2{\stdchapter[{#1}]{#2}}
\makeatother
 % -- -------------------------------------------------------------------------
 

 
 
%----------------------------------------------------------------------------------------
% Comandos del libro
%----------------------------------------------------------------------------------------
%OJO
%\usepackage{fontspec}
%\newcommand*{\eye}{{\fontspec{symbola.ttf}\symbol{"1F441}}}
%%
\newcommand{\dd}[2]{\dfrac{d #1}{d #2}\;}
\newcommand{\R}{\mathbb{R}}
\newcommand{\Z}{\mathbb{Z}}
\newcommand{\Q}{\mathbb{Q}}
\newcommand{\N}{\mathbb{N}}
\newcommand{\I}{\mathbb{I}}
\newcommand{\raya}{\rule{2cm}{0.01cm}\\}
\newcommand{\ds}{\displaystyle}
\newcommand{\sen}{\mathop{\rm sen}\nolimits}
\newcommand{\senh}{\mathop{\rm senh}\nolimits}
\newcommand{\arcsen}{\mathop{\rm arcsen}\nolimits}
\newcommand{\arcsec}{\mathop{\rm arcsec}\nolimits}
\newcommand{\bc}{\begin{center}}
\newcommand{\ec}{\end{center}}
\newcommand{\be}{\begin{enumerate}}
\newcommand{\ee}{\end{enumerate}}
\newcommand{\p}[1]{$\,#1\,$}
\newcommand{\gfrac}{\dfrac}
\newcommand{\dpr}[2]{\dfrac{\partial #1}{\partial #2}}
\newcommand{\dep}[2]{\,\dfrac{\partial #1}{\partial #2}\,}
\newcommand{\bv}[1]{\left.#1\right|}% uso \bv{}_P
\newcommand{\es}{\,\in\,}
\newcommand{\wred}[1]{{\red #1}}
\newcommand{\nC}{\pmb{C}}
\newcommand{\nN}{{\pmb{N}}}
\newcommand{\nF}{{\pmb{F}}}
\newcommand{\ndiv}{\mbox{\bf Div}\,}
\newcommand{\nrot}{\mbox{\bf Rot}\,}
\newcommand{\nni}{ \;\boldsymbol{\hat{\imath}}}
\newcommand{\nj}{  \;\boldsymbol{\hat{\jmath}}}
\newcommand{\nk}{  \;\boldsymbol{\hat{k}}}
\newcommand{\nr}{\pmb{r}}
\newcommand{\ndr}{\,\pmb{dr}}
\newcommand{\nS}{\,\pmb{S}}
\newcommand{\ndS}{\,\pmb{dS}}
\newcommand{\ent}{\;\Longrightarrow\;}
\newcommand{\imp}{\;\Longrightarrow\;}
\newcommand{\wmbox}[1]{\;\mbox{#1}\;}
\newcommand{\wstrut}{\rule[-.5\baselineskip]{0pt}{\baselineskip}}
%% Check marck
\usepackage{tikz}
\def\checkmark{\tikz\fill[scale=0.4](0,.35) -- (.25,0) -- (1,.7) -- (.25,.15) -- cycle;}
\def\wcheck{\tikz\fill[scale=0.4](0,.35) -- (.25,0) -- (1,.7) -- (.25,.15) -- cycle;}
 \newcommand{\wminipage}[2]{\begin{minipage}[t]{0.5\textwidth-0.25cm}
 #1
 \end{minipage}\hfill  \begin{minipage}[t][][b]{0.5\textwidth-0.25cm}
 #2
 \end{minipage}
 }

%-----------------------------------------------------------------------------------


\definecolor{styrmitcrtitlecajainteriorDefTeosc}{RGB}{14,44,142}
\newcommand{\wfnt}[1]{{\color{styrmitcrtitlecajainteriorDefTeosc} \fontfamily{pag}\fontsize{10}{1}\fontseries{b}\fontshape{n}\selectfont{#1}}}
\newcommand{\wsolu}{ {\wfnt{Solución:}}\;\black}
\newcommand{\exsolu}[1]{ { \azulf \wfnt{Solución.}\\ #1 }}
\newcommand{\Fdr}{\pmb{F}\cdot \,d\pmb{r}}
\newcommand{\FF}{\pmb{F}}
\newcommand{\FN}{\pmb{F}\cdot \pmb{N}}
 
 

\newcommandx\wparbox[4][1=9cm,2=9cm]{
\parbox{#1}{#3}\parbox{#2}{#4}
}
\newcommand{\intd}[2]{\displaystyle{\iint_{#1} #2\,dA}}

\newcommand{\dosfiguras}[2]{
\begin{minipage}{\textheight}
  \begin{minipage}[b]{.4\textheight}
    \bc
    #1
    \ec
  \end{minipage}%
  \begin{minipage}[b]{.4\textheight}
    \bc
    #2
    \ec
  \end{minipage}%
  \end{minipage}%
}


\newcommand{\figura}[2]{
\begin{tabular}{lc}
{\bf #1} & \\
          & #2\\
\end{tabular}
}

%\ndosfiguras{a}{fig 1}{b}{fig 2}
\newcommand{\ndosfiguras}[4]{
\bc
\begin{tabular}{ccc}
 \multicolumn{1}{l}{#1} & &\multicolumn{1}{l}{#3}\\
 #2 & & #4\\
\end{tabular}
\ec
}


\newcommand{\Texersol}[3]{\wparbox{
\exersol{ % primera columna
#1
}{% solucion
#3
} 
}{%------- segunda columna
#2
}
}

\newcommand{\tresfiguras}[3]{
\begin{minipage}{\textheight}
  \begin{minipage}[b]{.28\textheight}
    \bc
    #1
    \ec
  \end{minipage}%
  \begin{minipage}[b]{.28\textheight}
    \bc
    #2
    \ec
  \end{minipage}%
  \begin{minipage}[b]{.28\textheight}
    \bc
    #3
    \ec
  \end{minipage}%
  \end{minipage}%
  }

%----------------------------------------------------------------------------------------
% Copypright final p\'agina
\usepackage[absolute]{textpos}
%set unit to be pagewidth and height, and increase inner margin of box
\setlength{\TPHorizModule}{\paperwidth}\setlength{\TPVertModule}{\paperheight}
\TPMargin{3pt}
%define \copyrightstatement command for easier use
\newcommand{\finalpagina}{
    \begin{textblock}{0.84}(0.15,0.93)%{boxwidth}(leftposition,rightposition)
         \noindent
         \footnotesize
\copyright\fntg[phv][9]{\black {\it Revista Digital Matem\'atica, Educaci\'on e 
Internet.} \;
\href{https://tecdigital.tec.ac.cr/revistamatematica/Libros/}{
https://tecdigital.tec.ac.cr/revistamatematica/Libros/}}
    \end{textblock}
}

\newcommand{\info}[1]{
    \begin{textblock}{0.84}(0.1,0.93)%{boxwidth}(leftposition, rightposition)
         \noindent
         \footnotesize
\fntg[phv][9]{\verde #1}
    \end{textblock}
}


%----------------------------%enumeration----------------------------------
% \usepackage{chemgreek,textgreek} 
% \makeatletter
% % commands to format a counter value as Greek letter to be used like 
% % \arabic or \roman:
% \newcommand*\alphgreek[1]{\expandafter\@alphgreek\csname c@#1\endcsname}
% \newcommand*\@alphgreek[1]{\csname chemgreek_int_to_greek:n\endcsname{#1}}
% \newcommand*\Alphgreek[1]{\expandafter\@Alphgreek\csname c@#1\endcsname}
% \newcommand*\@Alphgreek[1]{\csname chemgreek_int_to_Greek:n\endcsname{#1}}
% 
% % register new counter formats to enumitem:
% \AddEnumerateCounter*{\alphgreek}{\@alphgreek}{\chemalpha}
% \AddEnumerateCounter*{\Alphgreek}{\@Alphgreek}{\chemAlpha}
\makeatother
%----------------------------------------------------------------------

% Aquí podría cambiar el color de los entornos------------------------
\newcommand{\bb}[1]{{\blue #1}}
\newcommand{\rr}[1]{{\red #1}}
\definecolor{colorejemplo}{RGB}{77,190,208}   % celestepastel
\definecolor{colordefinicion}{RGB}{104,48,39} % café
\definecolor{colorteorema}{RGB}{201,148,199}  % rosado pastel
\definecolor{colorcaja}{RGB}{160,128,104}     % café claro
\definecolor{colortitulo}{RGB}{0,133,202}
\definecolor{styrmitcrseudoindigo}{rgb}{0.345,0.1176,0.5490}
\newcommand{\clrv}{\color{styrmitcrseudoindigo}}
%\newcommand{\fl}{\longrightarrow}
\newcommand{\fl}{\pmb{\rightarrow}} %flec
%composition
\newcommand{\compcent}[1]{\vcenter{\hbox{$#1\circ$}}}
\newcommand{\comp}{\mathbin{\mathchoice
  {\compcent\scriptstyle}{\compcent\scriptstyle}
  {\compcent\scriptscriptstyle}{\compcent\scriptscriptstyle}}}
  \newcommand{\eq}{\;\pmb{\longleftrightarrow}\;}
%-----------------------------------------------------------------------
\usepackage{palatino}  % Fuente ppl buenas flechas, sin fourier
%----------------------------------------------------------------------
\usepackage{fourier} % usando fuentes "Fourier" en vez de "Palatino" 
% Insertar portada
\usepackage{pdfpages}

\makeatletter %contadores griegos
\newcommand{\greekalpha}[1]{\c@greekalpha{#1}}
\newcommand{\c@greekalpha}[1]{%
  {%
    \ifcase\number\value{#1} %
    \or
    \textalpha
    \or
    \textbeta
    \or
    \textgamma
    \or
    \textdelta
    \or
    \textepsilon
    \or
    \textzeta
    \or
    \texteta
    \or
    \texttheta % or \straighttheta
    \or
    \textiota
    \or
    \textkappa
    \or
    \textlambda
    \or
    \textmu
    \or
    \textnu
    \or
    \textxi
    \or
    \textomikron
    \or
    \textrho
    \or
    \textpi
    \or
    \textsigma
    \or
    \texttau
    \or
    \textupsilon
    \or
    \textphi
    \or
    \textchi
    \or
    \textpsi
    \or
    \textomega
    \fi
  }%
}
\AddEnumerateCounter*{\greekalpha}{\c@greekalpha}{5}
\makeatother

%%Referencia de secciones, capitulos etc

\newcommand{\y}{\boldsymbol{\;\wedge\;}}
\newcommand{\sii}{\,\Longleftrightarrow\,}
\newcommand{\llaves}[1]{\left\{#1 \right\}}
\newcommand{\paren}[1]{\left(#1 \right)}
\newcommand{\tq}{\wmbox{tal que}}
\newcommand{\warr}[2]{\begin{array}{#1}
#2
 \end{array}}
 \newcommand{\sys}[2]{\left\{\begin{array}{#1}
#2
 \end{array}\right.}
\newcommand{\wmt}[2]{\left[\begin{array}{#1}
#2
 \end{array}\right]}
%%CDFs
%% Link a los cdf
\newcommandx{\cfnte}[4][1=pag,2=9,3=n]{{\fontfamily{#1}\fontsize{#2}{1}\fontshape{#3}\selectfont{#4}}}



\newcommand{\Pp}{\mathcal P}
%\newcommand{\limite}[2]{\lim_{#1\fl #2}\;}

\newcommand{\colorazul}{}
\newcommand*\chancery{\fontfamily{pzc}\selectfont}
\newcommand{\set}[1]{\big\{#1\big\}}
%\renewcommand{\arraystretch}{1.5} % alto filas * 1.5

%\def\sen{\mathop{\mbox{\normalfont sen}}\nolimits}
%\newcommand{\tg}{\mathhop{\rm tg}\nolimits} %tangente
\newcommand{\g}{{}^\circ}
\renewcommand{\vector}[1]{\overrightarrow{#1}}
%\newcommand{\R}{\mathbb{R}}
%\newcommand{\Q}{\mathbb{Q}}
%\newcommand{\N}{\mathbb{N}}
%\newcommand{\Z}{\mathbb{Z}}
\newcommand{\rectaplanos}[2]{\left\lbrace \begin{array}{l} #1 \\
      #2 \end{array} \right.}

\newcommand{\recta}[6]{\dfrac{x#1}{#2}=\dfrac{y#3}{#4}=\dfrac{z#5}{#6}}
\newcommand{\rectap}[3]{\left\lbrace \begin{array}{l}#1 \\ #2 \\ #3 \end{array} \right. }
%Nuevos entornos
\newenvironment{matrizgauss}{\left( \begin{array}{ccc|ccc}}{\end{array} \right)}
\newenvironment{matrizampliada}{\left( \begin{array}{ccc|c}}{\end{array} \right)}
\newenvironment{sistematres}{\left\lbrace \begin{array}{rrrrrrr}}{\end{array} \right.}
\newenvironment{sistemados}{\left\lbrace \begin{array}{rrrr}}{\end{array} \right.}
\newenvironment{adjunto}{\[ \begin{array}{lll}}{\end{array} \]}
\newenvironment{sistema}{\left\lbrace \begin{array}{rrrrrr}}{\end{array} \right. }

%nuevos comandos

\newcommand{\limite}[2]{\displaystyle \lim_{x \rightarrow #1}{#2}}
\newcommand{\limiteserie}[1]{\displaystyle \lim_{n \rightarrow +\infty}{#1}}
\newcommand{\matrizdos}[4]{\begin{pmatrix} #1 & #2 \\  #3 & #4  \end{pmatrix}}
\newcommand{\determinantedos}[4]{\begin{vmatrix} #1 & #2 \\ #3 & #4  \end{vmatrix}}
\newcommand{\matriztres}[9]{\begin{pmatrix} #1 & #2 & #3 \\ #4 & #5 & #6 \\ #7 & #8 & #9 \end{pmatrix}}
\newcommand{\determinantetres}[9]{\begin{vmatrix} #1 & #2 & #3 \\ #4 & #5 & #6 \\ #7 & #8 & #9 \end{vmatrix}}
\newcommand{\filasistematres}[6]{#1 & #2 & #3 &#4 & #5 & = & #6}
\newcommand{\integral}[1]{\displaystyle \int #1}
\newcommand{\intdef}[3]{\displaystyle \int_#1^#2 #3}


%%Comandos para abreviar los entornos%%

\newcommand{\problema}[1]{\begin{ejer} #1  \end{ejer}}
\newcommand{\sol}[1]{\begin{solu} #1 \end{solu}}